\chapter{Conclusiones}\label{cap:conclusions}

Tras una breve introducción en el primer capítulo sobre los mesones $\PK$ y sus propiedades, en el segundo capítulo se ha explicado detalladamente como surgió el concepto de extrañeza (\cite{Nature1, Nishijima1955, Pais}) tras la primera observación de estas partículas. Los kaones tuvieron un papel clave a la hora de asentar los cimientos del Modelo de Quarks, permitiendo así, clasificar las partículas conocidas (hadrones) de acuerdo a sus propiedades y ordenarlas mediante los grupos de simetría. Además, se comprobó que la razón que había tras esta ordenación, era que dichas partículas poseían una estructura aún más fundamental, basada en combinaciones de quarks.

No obstante, experimentos recientes ($Q_{weak}$ \cite{nuruzzaman, carlini}), han demostrado que ciertas propiedades de las partículas, como la masa en el protón, no pueden explicarse únicamente mediante sus quarks de valencia y se requiere la existencia de un mar de quarks.

En el tercer capítulo se ha introducido en detalle el formalismo general de la interacción débil y su descripción en el contexto del Modelo de Quarks. Esto nos ha servido para definir las corrientes débiles tanto leptónicas como hadrónicas. Seguidamente, se ha llevado a cabo un estudio exhaustivo de los modos leptónicos de kaones cargados, donde se ha hecho uso del formalismo anterior junto con diagramas de Feynman para calcular la probabilidad de decaimiento del proceso $\PKm \rightarrow \Pl + \APnulepton$. Posteriormente, se han comparado los resultados con el decaimiento análogo del pión, concluyéndose que, para ambos mesones, el modo leptónico del electrón está mucho más suprimido que el del muón. Esto es debido a la conservación del momento angular y a la estructura tipo vector de la corriente débil. Por otro lado, la diferencia principal entre los llamados factores de forma $f_{\pi}$ y $f_{K}$ tiene que ver con el cambio de extrañeza que se produce en el modo leptónico del kaón.

Finalmente, en el cuarto capítulo, se explican las simetrías P y C y como gracias a los mesones $\PK$ se descubrió que ambas podían ser violadas (\cite{Wu1957}). Del mismo modo, el decaimiento de kaones neutros fue clave al descubrir la violación CP (\cite{Cronin}). Este fenómeno ha resultado esencial para intentar comprender la propia evolución del Universo. En la actualidad, uno de los campos de mayor interés en Física es el análisis de la violación CP en procesos con neutrinos, y su impacto en la asimetría materia-antimateria.
