\chapter*{Objetivos y metodología}
\addcontentsline{toc}{chapter}{Objetivos y metodología}\label{cap:objetivos}

La propuesta de este Trabajo de Fin de Grado surge de la gran motivación que supuso en la asignatura de Física Nuclear y Partículas la realización de un proyecto en grupo conocido como ``Adopta una Partícula'' en el cuál se escogió el mesón $K$ como partícula adoptada. Para nuestra sorpresa, esta partícula resultó ser de lo más fascinante.

Desde su descubrimiento, el mesón $K$ ha constituido un rol fundamental en la Física de Partículas. No sólo fueron las primeras partículas extrañas que se detectaron, sino que ello supuso una revolución total para la Física moderna: fueron los responsables de la introducción de la Extrañeza como nuevo número cuántico y ha servido de inspiración para sentar las bases del Modelo de Quarks y el hallazgo de cuatro de los seis quarks conocidos.

La teoría de Quarks ha tenido numerosas consecuencias de suma importancia en el estudio de las Interacciones Fundamentales y el Modelo Estándar. Gracias a ello ha sido posible predecir las posibles causas de violaciones de simetría, la existencia de nuevos quarks presentes en nuevas partículas y oscilaciones de sabor. Todo ello nos permite acercarnos un poco más a la compresión de la física de cortas distancias y el mundo microscópico. 

Por lo tanto, este trabajo pretende dar a conocer los mesones $K$ en profundidad y detallar todas estas implicaciones que su descubrimiento ha traído consigo, con el objetivo de concentrar toda esa información en un único documento, facilitando el trabajo de los divulgadores e investigadores de este campo de la Física que necesiten o, simplemente, quieran conocer más acerca de estas partículas extrañas tan útiles y curiosas.

Para tal fin, tras unas breves pinceladas sobre el contexto histórico, comenzaremos con un estudio en tono cualitativo de la Extrañeza y la definición de mesón $K$ en el Modelo de Quarks, seguido de un desarrollo más cuantitativo de la Interacción Débil, dónde haremos uso de su formalismo general. Finalmente, relacionaremos todo lo anterior con la violación de simetría CP y proporcionaremos algunos aspectos más actuales dónde se trabaja con mesones $K$.



