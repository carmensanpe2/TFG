\chapter{Extrañeza}\label{cap:strangeness}

Como se comentó anteriormente, los físicos británicos George Rochester y Clifford Butler, se hallaban tomando fotografías en una cámara de niebla en 1947, tratando de detectar partículas generadas por los rayos cósmicos al impactar sobre las moléculas de la atmósfera, cuando detectaron dos rastros en forma de $V$ invertida nunca observados hasta entonces, que sólo podían ser explicados por el decaimiento de una partícula neutra con masa entre 700 y 1600 veces mayor a la del electrón. 

En aquella época, ante la ausencia de aceleradores de partículas potentes, la búsqueda de partículas se realizaba estudiando los rayos cósmicos. Estos rayos no son más que radiación de alta energía procedente del espacio exterior. Cuando estos inciden en las distintas capas de la atmósfera producen una serie de reacciones en cadena que da lugar a una cascada de partículas. Además, a mayor altitud, mayor es el flujo de rayos cósmicos que nos llegan, lo que facilita que un mayor número de partículas llegue al detector. Por este motivo, Rochester y Butler decidieron realizar nuevamente el experimento en los Pirineos Franceses, concretamente en el Observatorio \textit{Pic Du Midi}, y esta vez consiguieron detectar decenas de estas nuevas partículas. En un principio se las denominó partículas $V$ pero posteriormente pasaron a identificarse como mesones $K$ o kaones. \cite{Griffiths2008}

Para entender cómo lograron detectar los mesones $K$ por primera vez, hay que entender el funcionamiento de las cámaras de niebla. En estos dispositivos son entornos cerrados donde se llevaba a cabo el estudio de los rayos cósmicos: las partículas que componen los rayos atraviesan un volumen con vapor de agua sobresaturado contenido en dicho recinto cerrado. Las partículas con carga eléctrica producen una cierta ionización de aire provocando la condensación del vapor de agua a lo largo de la trayectoria.  Al situar la cámara de niebla en campos eléctricos y magnéticos, y estudiando las curvaturas de las trayectorias es posible obtener la carga eléctrica, la energía y la masa de las partículas.  Asimismo, como muchas partículas decaen en otras por ser inestables, estudiando la longitud de las trazas que dejan las partículas al decaer en la cámara de niebla, se puede deducir su vida media. \cite{notas2020}\\

En diciembre de 1947, se publicaron en la revista \textit{Nature} dos de las numerosas fotografías que habían tomado Rochester y Butler, las cuales podemos observar a continuación:



\begin{figure}[h]
	\centering
	\includegraphics[width=0.8\textwidth]{C:/Users/Carmen/Desktop/Universidad/TFG/Borradores/img/rochester1.PNG}
	\caption[Fotografía 1 de la primera detección de los mesones $K$]
	{Primera fotografía estereoscópica publicada en la revista \textit{Nature}. \cite{Nature1}}
	\label{fig:nature1}
\end{figure}

\begin{figure}[h]
	\centering
	\includegraphics[width=0.8\textwidth]{C:/Users/Carmen/Desktop/Universidad/TFG/Borradores/img/rochester2.PNG}
	\caption[Fotografía 2 de la primera detección de los mesones $K$]
	{Segunda fotografía estereoscópica publicada en la revista \textit{Nature}. \cite{Nature1}}
	\label{fig:nature2}
\end{figure}


En la figura \ref{fig:nature1}, se observa como las partículas de rayos cósmicos entran por la parte superior izquierda y colisionan contra la placa de plomo, produciendo una partícula neutra, cuya presencia se hace evidente al decaer en otras partículas cargadas más ligeras, formando una ``V'' invertida en la parte inferior derecha, señalada mediante las marcas a y b. La figura \ref{fig:nature2} muestra un proceso similar pero esa vez, se produce una nueva partícula cargada (por entonces partícula-$\tau$) que decae en otras dos partículas más ligeras, una cargada y la otra neutra, apreciándose una desviación o ``kink'' en las trayectorias.\\

%Para entender las fotografías en mayor detalle, a continuación se muestra un esquema de uno de los procesos.

%\begin{figure}[h]
%	\centering
%	\includegraphics[width=0.8\textwidth]{C:/Users/Carmen/Desktop/Universidad/TFG/Borradores/img/esquema.PNG}
%	\caption[Esquema para entender las fotos estereoscópicas]
%	{Esquema del proceso observado en la primera fotografía estereoscópica. \cite{Griffiths2008}}
%	\label{fig:esquema1}
%\end{figure}

Años más tarde se comprobó que la primera fotografía \ref{fig:nature1} correspondía al proceso  $K^0 \rightarrow \pi^+$ +  $\pi^- $ y la segunda fotografía (\ref{fig:nature2}) a la reacción $K^+ \rightarrow \mu^+$ + $\nu $.

El comienzo de los años 50 trajo consigo técnicas innovadoras de detección de partículas, como la cámara de burbujas, detectores de emulsión nuclear muy precisos y nuevos aceleradores más potentes. Estos nuevos dispositivos demostraron un gran avance tecnológico al permitir la producción casi en masa de las partículas descubiertas en la década anterior. 

De hecho, los mesones $K$ habían quedado un poco en el olvido durante los dos años posteriores a su descubrimiento hasta que en 1950 otros célebres físicos de partículas como Powell, empezaron a publicar trabajos donde también se apreciaban los trazos en forma de $V$ y las desviaciones, que evidenciaban la presencia de las partículas $V$ y $\tau$, respectivamente. Coincidían con Rochester y Butler en que estos eventos representaban el decaimiento espontáneo de partículas neutras y cargadas, desconocidas hasta entonces. 

\begin{figure}[h]
	\centering
	\includegraphics[width=0.5\textwidth]{C:/Users/Carmen/Desktop/Universidad/TFG/Borradores/img/powell.PNG}
	\caption[Fotografía de Powell mostrando una trayectoria ``kink'']
	{Fotografía publicada por el grupo de Powell en la revista \textit{Nature}. \cite{Griffiths2008}}
	\label{fig:powell}
\end{figure}

El proceso descubierto por Powell en \ref{fig:powell} mostraba como $K^+$ entrando desde arriba decae en el punto A en tres piones cargados, dos positivos y uno negativo. El $\pi^-$ provoca seguidamente una desintegración en B. El proceso descrito era: $K^+ \rightarrow \pi^+$ + $\pi^+$ +  $\pi^- $.

Sin embargo, la partícula $K^0$ primero fue conocida como $V^0$ y luego como $\theta^0$ antes de renombrarse finalmente como $K^0$, mientras que la partícula $K^+$ se denominó en un principio como $\tau^+$. Esto es debido a que los procesos de \ref{fig:nature1} y \ref{fig:powell} tenían estados finales con distinta paridad y por eso se pensaba que debía tratarse de partículas diferentes.

$\theta^0$ y $\tau^+$ no se identificaron como distintas versiones de una misma partícula hasta 1956, cuando se resolvió el conocido enigma $\tau$-$\theta$ al descubrir la violación de la paridad en la interacción débil, de la cual hablaremos más adelante. \cite{Griffiths2008}

En los años siguientes empezaron a clasificar las partículas $V$ y $\tau$ en función de la masa. Así, se determinó que los mesones $K$ correspondían a aquellas partículas con masa intermedia entre el pión y el protón, mientras que aquellas con masas intermedias entre el neutrón y el deuterón pasarían a conocerse como hiperones. 

Volviendo al inicio de los años 50, fue también entonces cuando empezaron hacerse notar los fenómenos inusuales que presentaban estas partículas y que fueron los responsables de que se las bautizara como partículas extrañas, pues nunca se habían manifestado en otras partículas. Las partículas extrañas incluyen no sólo los mesones $K$, sino también los hiperones $\Lambda$, $\Sigma$ y cascada $\Xi$, entre otros. Pero para lo que nos concierne en este documento, vamos a centrar nuestro estudio en los kaones.

La primera característica extraña, mencionada ya antes varias veces, es el hecho de que estas partículas, a pesar de producirse por interacción fuerte de forma muy numerosa, tenían vidas medias relativamente largas, por lo que se sospechaba que decaían mediante interacción débil. El otro suceso curioso es el de producción asociada, que consiste en que las partículas extrañas sólo se producen por pares.  

Tras muchos intentos fallidos de dar explicación a estos hechos, el físico holandés-estadounidense Abraham Pais sugirió, en 1952, que las partículas elementales debían tener una nueva propiedad o regla de selección, a la que denominó ``Regla par-impar''. Esta regla de selección debía de cumplirse siempre para la Interacción Fuerte y la Electromagnética, pero podía violarse en la Interacción Débil. A grandes rasgos, la regla consistía en asignar el número 0 a las partículas ``antiguas'' ($\pi$, nucleones, $\gamma$, leptones) y el número 1 a las partículas ``nuevas'' $K$ y $\Lambda$ (el resto de las partículas todavía se desconocían). Dado cualquier proceso, se suman los números asignados de las partículas en el estado inicial y luego las del estado final. En la interacción fuerte y electromagnética, el número total inicial y final deben ser ambos pares o ambos impares, mientras que, en la interacción débil, uno debe ser par y el otro impar. Esta propiedad también daba respuesta a por qué las partículas extrañas debían producirse en pares. \cite{Pais}

La evidencia experimental de la idea de Pais vino de la mano de Murray Gell-Mann y Kazuhiko Nishijima en 1953. Por un lado, indicaban que la ``Regla par-impar'' debía ser una consecuencia directa de la independencia de carga de las partículas $V$, que indicaba que la carga de estas partículas podía encontrarse en tres estados: neutro, positivo y negativo. Por otro lado, Gell-Mann propuso asignar un número entero de Isospín $I$ a los hiperones y un semientero a los mesones $K$. Nakano y Nishijima también hicieron esta misma propuesta casi paralelamente. Dependiendo de si el isospín $I$ y su tercera componente $I_3$ se conservan o no, podrá llevarse acabo la producción de partículas mediante Interacción Fuerte. \cite{nakano}

En 1955, Nishijima planteó, a partir de sus observaciones experimentales, que los mesones $K$ se podían organizar en dobletes de carga y, por tanto, podían obtenerse partículas $K$ con carga conjugada. $K^+$ y $K^0$ formaban un doblete de carga con $I=1/2$ e $I_3=1/2$ y $-1/2$ respectivamente y, además, ambas partículas podían definirse como funciones de onda complejas. Ello permitía establecer el doblete conjugado de sus antipartículas $K^-$ y $\overline{\rm K^0}$, con $I_3=-1/2$ y $1/2$ respectivamente, relacionados entre sí mediante el operador conjugación de carga. Gell-Mann también llegó a conclusiones similares en 1956. Profundizaremos en mayor medida sobre este tema en los próximos capítulos. \cite{Nishijima1955}

Por otro lado, se comprobó que en todas las interacciones antes descritas se conservaba el número bariónico $B$ (si bien no se descarta en un futuro alguna excepción)\footnote{ver sección \ref{cap:non-strange_particles}}. Asimismo, desarrollando toda esta teoría de la invariancia de carga para las partículas $V$ y para incluir la propiedad sugerida por Pais, se introdujo un nuevo número cuántico al que Gell-Mann denominó \textit{extrañeza} o \textit{Strangeness} $S$ (Nishijima lo denotaba como carga-$\eta$). Asignaron $S=1$ al doblete de partículas de mesones $K$ y $S=-1$ para el doblete de sus antipartículas. Este nuevo número cuántico resultaba últil para relacionar la carga con $I_3$ en partículas elementales, llegándose a una expresión conocida como la \textit{Fórmula de Gell-Mann-Nishijima}:

\begin{equation}
Q=I_3+ \frac{B+S}{2}=I_3+\frac{Y}{2}
\end{equation}

Siendo $Y$ la \textit{hipercarga}, definida como $Y=B+S$ en un principio.

El teorema de Conservación de la Extrañeza se formuló a partir de las evidencias experimentales: $S$ debía conservarse en las Interacciones Fuerte y Electromagnética, pero podía ser violada en procesos ocurridos por Interacción Débil, tal y como había sugerido Pais con su \textit{Regla par-impar}.


\section{Extrañeza en el Modelo de Quark}\label{cap:strangeness_quark_model}
\vspace{5mm}
Como hemos visto, el concepto de extrañeza se introdujo por motivos puramente fenomenológicos, gracias a las observaciones experimentales. Sin embargo, a través del Modelo de Quarks, la extrañeza obtiene su base teórica.

A principios de la década de los 60, y tras el descubrimiento de muchas partículas ``elementales'' se hizo necesario poner cierto orden haciendo uso de las simetrías. Las simetrías son útiles porque permiten clasificar de forma ordenada las partículas en función de sus propiedades. La rama de las matemáticas encargada de tratar con simetrías es la teoría de grupos. El isospín viene descrito por el grupo $SU(2)$, pero el descubrimiento de las partículas extrañas hizo necesario extender el concepto de isospín a una simetría superior. Además, había ciertas propiedades que sugerían la posibilidad de que, en vez de ser elementales, las partículas tuvieran una estructura interna compleja. 

Así pues, Gell-Mann y Ne'eman observaron que los valores de isospín y la hipercarga de los hadrones estaban relacionados entre sí, tal que los hadrones (mesones y bariones) aparecían en grupos, con el mismo espín y paridad y masas similares, denominados singletes (una partícula), octetes (ocho partículas) y decupletes (diez partículas). Esta descripción concordaba con la estructura del grupo de simetría interna $SU(3)$ si fundamentaban las transformaciones en la independencia de carga de la interacción fuerte (isospín) y en la conservación de la extrañeza y, además, suponía que las partículas estaban constituidas por entes más fundamentales, a los que denominaron \textit{quarks}. \cite{notas2020}
Las diversas representaciones del $SU(3)$, no sólo permitían la clasificación de las partículas en función de sus números cuánticos, sino que además, predecían la existencia de nuevas partículas aún por descubrir. 

En 1964, Gell-Mann y Zweig indicaron que

En concreto, los mesones $K$ se representaban junto a los piones en un hexágono que los clasifica según sus números cuánticos de carga $Q$ y extrañeza $S$, formando un octete pseudo-escalar. Dicho octete está formado por el triplete de los piones y los dos dupletes de las partículas $K$ y sus antipartículas, aunando así todos los mesones ligeros en un mismo grupo. En la siguiente figura se puede ver la representación de este octete:

\begin{figure}[h]
	\centering
	\includegraphics[width=0.7\textwidth]{C:/Users/Carmen/Desktop/Universidad/TFG/Borradores/img/octete.PNG}
	\caption[Octete de mesones ligeros]
	{Octete de mesones ligeros. \cite{Griffiths2008}}
	\label{fig:octete}
\end{figure}



\section{Extrañeza en partículas <<no extrañas>>}
\label{cap:non-strange_particles}
El protón