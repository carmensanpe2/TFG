\chapter{Violación CP}\label{cap:CP_violation}
Los conceptos de paridad y violación de paridad han sido utilizados varias veces en los capítulos anteriores, pero no hemos explicado en qué consisten en ningún momento. Antes de comenzar con la descripción de los mesones $\PK$ neutros y la violación CP, resulta útil dar unas pinceladas sobre el concepto de paridad y qué papel jugaron los kaones en el descubrimiento de su violación en la fuerza débil. 

Cuando ciertas propiedades de una partícula no cambian al someterla a un conjunto de transformaciones, se dice que esa partícula tiene una simetría. De acuerdo con el Teorema de Noether, cada simetría se asocia a una magnitud física que se conserva, las cuales se describen mediante operadores hermíticos. Hay dos simetrías discretas importantes que conviene discutir en relación con los mesones $\PK$, ya que en la interacción fuerte y en la
electromagnética se conservan, pero pueden violarse en la interacción débil. Estas dos simetrías son la inversión espacial P y la conjugación de carga C.

En la invariancia frente a la inversión espacial o simetría P (a veces, paridad) se invierte el signo de las coordenadas de las partículas. En consecuencia, los vectores polares, como la posición $\vec{r}$ y el momento lineal $\vec{p}$, cambian su signo, mientras que los vectores axiales, como el momento angular orbital $\vec{l}$, lo conservan. Esta simetría se describe mediante el operador paridad $\hat{P}$. La paridad intrínseca o paridad-P $\eta _{P}\left(A \right)$ se determina empíricamente y se asocia con la estructura interna de la partícula A, pudiendo ser su valores $\pm 1$. Para fermiones y bosones se tiene $\eta _{P}\left(f \right)= -\eta_{P} \left(\overline{f} \right)$ y $\eta_{P}\left(b \right)=\eta_{P}\left(\overline{b}\right)$, respectivamente. Además, se cumple que $[\hat{P}, \widehat{H}]=0$ \cite{notas2020}. 

De forma similar, la invariancia frente a la conjugación de carga o simetría C, transforma partículas A en antipartículas $\overline{A}$ y viceversa, de modo que cambia el signo de la carga y del resto de números cuánticos aditivos (bariónico $B$, leptónico $L$, etc.) El operador asociado es $\hat{C}$. Nuevamente, $\eta _{C}\left(A\right)= \pm 1$ y $[\hat{C} , \widehat{H}]=0$. En este caso, sólo las partículas neutras tienen la paridad-C  $\eta _{C}\left(A \right)$ bien definida \cite{notas2020}. 

Estas dos simetrías parecen conservarse siempre en los procesos donde intervenían interacciones fundamentales. Sin embargo, el hallazgo de los mesones $\PK$ trajo consigo las primeras evidencias de que la interacción débil podía violar las simetrías P y C.


\subsubsection{Enigma $\Ptau$-$\theta$}
Tras las primeras observaciones de los mesones $\PK$ en los años 50, había un par decaimientos de estas, por entonces llamadas, partículas $V$ que desconcertaba a los científicos de la época. Tanta era la confusión que incluso se llegó a pensar que había dos tipos de partículas $V$, a las que denominaron $\Ptau$ y $\theta$, como se mencionó en el capítulo \ref{cap:strangeness}. 

Recordemos que, por una parte, en la fotografía \ref{fig:nature1}, se observaba el decaimiento $\theta^{0} \rightarrow \Ppiplus\Ppiminus$ y años más tarde también se había constatado la existencia del proceso $\theta^{+} \rightarrow \Ppiplus\Pgpz$. Por otro lado, la fotografía \ref{fig:powell} mostraba la desintegración de $\APtauon \rightarrow \Pgpp\Pgpp\Pgpm$.  El estudio de estos dos procesos concluía que las partículas $\Ptau$ y $\theta$ eran idénticas pues tenían las mismas masas, las mismas vidas medias, etc., pero dado que los decaimientos tenían distinta paridad, era \textit{imposible} que fueran las mismas partículas; no se concebía que la paridad fuera violada. Esto hecho fue conocido como el \textit{enigma $\Ptau$-$\theta$} \cite{Ferbel}.
\begin{equation}
\begin{gathered}
\theta^{+} \rightarrow \pi^{+}\pi^{0} \Rightarrow \eta\left(\pi^{+}\right) \eta\left(\pi^{+}\right) = 1 \\ 
\tau^{+} \rightarrow \pi^{+}\pi^{+}\pi^{-} \Rightarrow \eta\left(\pi^{+}\right) \eta\left(\pi^{+}\right) \eta\left(\pi^{-}\right) = -1
\end{gathered}
\end{equation}
Lee y Yang estudiaron a fondo el rompecabezas, concluyendo que no había ninguna evidencia para afirmar que la paridad se conservara en la interacción débil, por lo que las partículas $\tau$ y $\theta$ debían ser la misma y la violación de la paridad en esta interacción era una realidad más que plausible. 
El experimento clave que corroboró esta hipótesis, realizado por Wu y Ambler, consistió en analizar la emisión $\beta$ del $\tensor*[^{60}]{\mathrm{Co}}{}$ polarizado.


\section{Teoría Electrodébil}\label{sec:electroweak}
Describir unificación de la interacción débil y la electromagnética
e introducir formalismo $Z^0$
\section{Decaimiento de mesones $K$ neutros}
\label{sec:neutral_kaon_decay}

Explicar K-long y K-short, por qué se introducieron y principales decays

\subsection{Oscilaciones de mesones K}\label{sec:kaon_oscillations}

A raíz de lo anterior, explicar las oscilaciones de kaones