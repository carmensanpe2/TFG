\appendix
\chapter*{Apéndice A}\label{cap:A}
\addcontentsline{toc}{chapter}{Apéndice A}
\setcounter{section}{0}
\renewcommand{\thesection}{A.\arabic{section}}
\setcounter{table}{0}
\renewcommand{\thetable}{A.\arabic{table}}
\setcounter{equation}{0}
\renewcommand{\theequation}{A.\arabic{equation}}

\section{Transformaciones de Lorentz y Cuadrivectores}\label{sec:Lorentz}
Las \textit{Transformaciones de Lorentz} (LT) son las expresiones matemáticas encargadas de relacionar un suceso observado desde dos sistemas de referencia inerciales distintos, en base a los postulados de la Teoría de Relatividad Especial. Son las siguientes:
\setlength{\abovedisplayskip}{6pt}
\setlength{\belowdisplayskip}{6pt}
\begin{align}
x' &= \dfrac{x-vt}{\sqrt{1-v^2/c^2}} & y' &=y & z' &=z & t' &=\dfrac{t-xv/c^2}{\sqrt{1-v^2/c^2}}\label{eq:TLorentz1}
\end{align}

La forma más sencilla de expresar las LT es utilizando los cuadrivectores: vectores de cuatro componentes que permiten apreciar fácilmente qué magnitudes son escalares, es decir, invariantes frente a las LT (formulación covariante) \cite{MCR}.
\begin{align}
(ct,x, y, z) &=(x^0, x^1, x^2, x^3)=X^\mu & (E,cp_x, cp_y, cp_z) &=(p^0, p^1, p^2, p^3)=P^\mu
\end{align}
Hay dos tipos de cuadrivectores:
\begin{itemize}
\item Contravariante (índice arriba): $X^\mu=(x^0, x^1, x^2, x^3)=(x^0,\textbf{x}) \rightarrow (t,\vec{\boldsymbol{x}})$
\item Covariante (índice abajo): $X_\mu=(x_0, x_1, x_2, x_3)=(x_0,-\textbf{x}) \rightarrow (t,-\vec{\boldsymbol{x}})$
\end{itemize} 
Para relacionarlos, se usa el tensor métrico $g_{\mu\nu}$ que define la métrica del espacio-tiempo de Minkowski \cite{MCR}.
\begin{align}
A_{\mu } &=\sum ^{3}_{\nu =0}g_{\mu \nu }A^{\nu} \equiv g_{\mu \nu }A^{\nu} & A^{\mu } &=\sum ^{3}_{\nu =0}g^{\mu \nu }A_{\nu} \equiv g^{\mu \nu }A_{\nu} \label{eq:metrica}
\end{align}
con $g_{\mu\nu} = g^{\mu\nu}$, siendo:
\begin{equation}
g_{\mu\nu} = 
\begingroup 
\renewcommand*{\arraystretch}{0.8}
\setlength\arraycolsep{8pt}
\begin{pmatrix}
1 & & &  \\
& -1 & & \\
& & -1 & \\
& & & -1
\end{pmatrix}
\endgroup
\end{equation}
Además, se cumple lo siguiente:
\begin{itemize}
\item Norma de un cuadrivector $X^\mu$: $X^{2}=X^{\mu }X_{\mu }=X_{\mu }X^{\mu }=x_{0}^{2}-\boldsymbol{x}^{2}$
\item Producto escalar de dos cuadrivectores $A^{\mu}=\left( a^{0},\boldsymbol{a}\right)$ y $B^{\mu}=\left( b_{0},\boldsymbol{b}\right)$: $A\cdot B=A^{\mu }B_{\mu }=A_{\mu }B^{\mu }=a_{0}b^{0}- \boldsymbol{a}\cdot \boldsymbol{b}$. El escalar resultante se dice que es un invariante de Lorentz.
\end{itemize}

Por lo tanto, las LT para las componentes contravariantes y covariantes, respectivamente, pueden expresarse \cite{MCR} tal que así:
\begin{align}
X'^{\mu } &=\Lambda\indices{^\mu _\nu}X^\nu & X'_{\mu } &=\Lambda\indices{_\mu ^\nu}X_\nu \label{eq:LTcovariant}
\end{align}
En general, $\Lambda\indices{^\mu _\nu} \neq \Lambda\indices{_\mu ^\nu}$.  Gracias a todo esto, se pueden definir los siguientes operadores:
\begin{align}
\partial _{\mu} &= \dfrac{\partial}{\partial x^{\mu}}=\left( \dfrac{\partial }{c\partial t},\vec{\nabla} \right) & \partial ^{\mu } &= \dfrac{\partial }{\partial x_{\mu }}=\left( \dfrac{\partial }{c\partial t},-\vec{\nabla} \right) & \partial ^{\mu }\partial _{\mu }&= \dfrac{\partial ^{2}}{c^{2}\partial t^{2}}-\nabla ^{2} \equiv\square
\end{align}
\vspace{5mm}

\section{Ecuación de Dirac: Matrices y espinores}\label{sec:Dirac}
La ecuación de Dirac es una función de onda relativista que permite describir partículas fermiónicas ($s=1/2$). En el sistema natural de unidades $c=\hbar =1$, tiene esta expresión:
\begin{equation}
i\partial _{t}\Psi = -i\left( \alpha _{1}\dfrac{\partial }{\partial x^{1}}+\alpha _{2}\dfrac{\partial }{\partial x^{2}} + \alpha _{3}\dfrac{\partial }{\partial x^{3}}\right) \Psi +\beta m\Psi \equiv\widehat{H}\Psi \label{eq:Dirac}
\end{equation}
Los coeficientes $\alpha_i$ y $\beta$ son matrices. En la base de Dirac-Pauli\footnote{La representación de Dirac-Pauli es la más común, pero otras representaciones son posibles, como la que hace uso de la base de Weyl.} dichas matrices son:
\begin{align}
\alpha _{i}=
\begingroup 
\renewcommand*{\arraystretch}{0.8}
\setlength\arraycolsep{10pt}
\begin{pmatrix} 
0 & \boldsymbol{\sigma_{i}} \\ \boldsymbol{\sigma_{i}} & 0 \end{pmatrix} \qquad
\beta =\begin{pmatrix} \mathbb{1} & 0 \\ 0 & -\mathbb{1} \end{pmatrix}
\endgroup
\end{align}
siendo las \boldsymbol{$\sigma_{i}$} las matrices de Pauli:
\begin{align}
\sigma _{x}=
\begingroup 
\renewcommand*{\arraystretch}{0.8}
\setlength\arraycolsep{10pt}
\begin{pmatrix} 
0 & 1 \\ 1 & 0 \end{pmatrix} \qquad \sigma_y
\begin{pmatrix} 
0 & -i \\ i & 0 \end{pmatrix} \qquad
\sigma_z =\begin{pmatrix} 1 & 0 \\ 0 & -1 \end{pmatrix}
\endgroup
\end{align}

A partir de los coeficientes anteriores y utilizando notación en cuadrivectores, se obtienen las \textit{Matrices de Dirac} o \textit{Matrices Gamma}  $\gamma^{\mu}=(\gamma^0, \boldsymbol{\gamma^{i}})=(\gamma^0, \gamma^1, \gamma^2, \gamma^3)$:
\begin{align}
\gamma ^0 &\equiv \beta & \gamma ^i &\equiv \beta \alpha_i \hspace{1cm} i=1, 2, 3 \label{eq:matricesDirac}
\end{align}
Las matrices de Dirac $\gamma^{\mu}$ satisfacen la relación de anticonmutación \cite{MCR}:
\begin{equation}
\left\{ \gamma ^{\mu },\gamma ^{\nu }\right\} =\gamma ^{\mu }\gamma ^{\nu }+\gamma ^{\nu }\gamma ^{\mu }=2g^{\mu \nu }\mathbb{1}\label{eq:anticomm_relation}
\end{equation}
\begin{itemize}
\item $\gamma^{i}$ es unitaria y antihermítica: $\left( \gamma ^{i}\right) ^{-1}=\left( \gamma ^{i}\right) ^{\dagger}=-\gamma ^{i}$
\item $\gamma^{0}$ es unitaria y hermítica: $\left( \gamma ^{0}\right) ^{-1}=\left( \gamma ^{0}\right) ^{\dagger}=\gamma ^{0}$
\end{itemize}

Entonces la ecuación de Dirac, sabiendo que $\beta^2=\mathbb{1}$, se reformula como \cite{MCR}:

\begin{equation}
i\left( \gamma ^{0}\dfrac{\partial }{\partial t}+\sum ^{3}_{k=1}\gamma ^{k}\dfrac{\partial }{\partial x^{k}}\right) \Psi -m\Psi =0\label{eq:ecDirac_cov}
\end{equation}

Usando la notación abreviada conocida como \textit{notación ``slash'' de Feynman} \cite{MCR}, definimos el siguiente operador:
\begin{equation}
\slashed{\partial}=\gamma_{\mu }\partial ^{\mu }=\gamma ^{\mu }\partial _{\mu }=\gamma ^{0}\partial _{t}+\gamma \nabla\label{eq:slashnot}
\end{equation}
Por lo que la ecuación de Dirac resulta:
\begin{equation}
\left(i \slashed{\partial}-m\right) \Psi =0\label{eq:Dirac_final}
\end{equation}
Asimismo, para partículas libres, usando que $\widehat{p}^{\mu} \rightarrow i\partial^{\mu} \Rightarrow \slashed{\widehat{p}} \rightarrow i\slashed{\partial}$ \cite{MCR}, la ecuación de Dirac queda:
\begin{equation}
\left( \slashed{\widehat{p}}-m\right) \Psi =0\label{eq:Dirac_free}
\end{equation}

Las soluciones de esta ecuación tienen forma de onda plana y $\Psi$ se representan como autovectores de 4 componentes o bi-espinores, conocidos como \textit{espinores de Dirac} \cite{MCR} y, a su vez, cada bi-espinor puede expresarse en términos de dos espinores de 2 componentes:
\begin{align}
\begingroup 
\renewcommand*{\arraystretch}{0.8}
\setlength\arraycolsep{10pt}
\Psi \left( \boldsymbol{x}, t\right) =\begin{pmatrix} \psi_{1} \\ \psi_{2} \\ \psi_{3} \\ \psi_{4} \end{pmatrix}=\begin{pmatrix} u(\boldsymbol{p},s) \\ v(\boldsymbol{p},s) \end{pmatrix}; \qquad u(\boldsymbol{p},s)=\begin{pmatrix} \psi_{1} \\ \psi_{2} \end{pmatrix} \qquad
v(\boldsymbol{p},s)=\begin{pmatrix} \psi_{3} \\ \psi_{4}\label{eq:espinores} \end{pmatrix}
\endgroup
\end{align}
El espinor $u(\boldsymbol{p},s)$ corresponde a la solución con valores de energía positivos, que describen partículas, mientras que el espinor $v(\boldsymbol{p},s)$ está asociado a valores negativos de la energía, relacionados con las antipartículas \cite{Donelly}. Cada espinor presenta dos componentes, correspondientes a los dos posibles estados de la tercera componente de espín $s_z=\pm 1/2$. \cite{Bettini} Las soluciones conjugadas se definen como $\overline{u} \equiv u^{\dagger} \gamma^{0}$ y $\overline{v} \equiv v^{\dagger} \gamma^{0}$, respectivamente.

Los espinores $u(\boldsymbol{p},s)$ y $v(\boldsymbol{p},s)$ y sus conjugados cumplen la ecuación de Dirac \cite{Donelly}:
\begin{equation}
\begin{aligned}
(\slashed{p}-m)u(\boldsymbol{p},s) &=0 & (\slashed{p}+m)v(\boldsymbol{p},s) &=0 \\
\overline{u}(\boldsymbol{p},s)(\slashed{p}-m) &=0 & \overline{v}(\boldsymbol{p},s )(\slashed{p}+m)&=0
\end{aligned}
\end{equation}
La normalización de espinores es \cite{MCR}:
\begin{align}
\overline{u}(\boldsymbol{p},s)u(\boldsymbol{p},s) &= 1 & \overline{v}(\boldsymbol{p},s)v(\boldsymbol{p},s) &= -1
\end{align}
Y la relación de completitud \cite{MCR}:  %a \cite{Paschos}:
\begin{equation}
\sum _{s}\left| u_{\alpha }\left(\boldsymbol{p},s\right) \overline{u}_{\beta }\left(\boldsymbol{p},s\right) -v_{\alpha }\left(\boldsymbol{p},s\right) \overline{v}_{\beta }\left(\boldsymbol{p},s\right) \right| =\delta _{\alpha \beta }
\end{equation}

Además de las cuatro matrices gamma, existe una quinta matriz importante $\gamma^5$, cuya expresión en la representación de Dirac es:
\begin{equation}
\gamma^5 = i \gamma^0 \gamma^1 \gamma^2 \gamma^3\label{eq:gamma5}
\end{equation}
y cumple que: $\gamma^5 \gamma^5 = 1$.

\subsubsection{Covariantes bilineales}\label{sec:bilinearcov}
Los covariantes bilineales $\widehat{\Gamma}$ son muy útiles para la descripción de un lagrangiano de interacción, al proporcionar información sobre cómo se transforman los espinores. Son los siguientes:

\begin{table}[h]
	\centering
	\begin{tabular}{l*{8}{c}r}
\hline
Expresiones de $\widehat{\Gamma}$  & Carácter & Nº. Matrices\\ 
\hline
$\widehat{\Gamma}^{S} = \mathbb{1}$ & Escalar (S) & 1\\
$\widehat{\Gamma}_{\mu}^{V} = \gamma_{\mu}$ & Vectorial (V) & 4 &\\
$\widehat{\Gamma}^{T}_{\mu\nu} = \hat{\sigma}_{\mu\nu}-\hat{\sigma}_{\nu\mu}$ & Tensorial (T) & 6\\
$\widehat{\Gamma}_{\mu}^{A} = \gamma_{5}\gamma_{\mu}$ & Vector axial (A) & 4\\
$\widehat{\Gamma}^{P} = \gamma_{5}$ & Pseudo-vector (P) & 1\\
\hline
	\end{tabular}
\caption[Covariantes bilineales]{Covariantes bilineales. \cite{MCR}, \cite{GreinerRQM}}
\label{tab:bilinear_covariant}
\end{table}

con $\hat{\sigma}^{\mu\nu}= \dfrac{i}{2} \left(\gamma^{\mu}\gamma^{\nu} - \gamma^{\nu}\gamma^{\mu} \right)$.

%\sum_s u(\boldsymbol{p},s)\overline{u}(\boldsymbol{p},s) &= (\slashed{p}+m) & \sum_s v(\boldsymbol{p},s)\overline{v}(\boldsymbol{p},s) &= (\slashed{p}-m)
\subsubsection{El truco de Casimir y los teoremas de las trazas}\label{sec:trazas}
El truco de Casimir es útil a la hora de calcular la amplitud media $\left\langle|\mathcal{M}|^2\right\rangle$ de un proceso de decaimiento haciendo uso de las trazas de las matrices.
\begin{equation}
\sum _{spins}\left[ \overline{u}\left( a\right) \Gamma_{1}u\left( b\right) \right] \left[ \overline{u}\left( a\right) \Gamma _{2}u\left( b\right) \right] ^{\ast }= \Tr\left[ \Gamma _{1}\left( \slashed{p_{b}}+m_{b}c\right) \cdot \overline{\Gamma }_{2}\left( \slashed{p_{a}}+m_{c}c\right) \right]\label{eq:casimir_trick}
\end{equation}

Una demostración detallada de esta herramienta matemática puede encontrarse en la sección 7.7 de la referencia \cite{Griffiths2008}.

Las principales identidades de las matrices y los teoremas de las trazas son los siguientes:

\begin{enumerate}
\setlength{\itemsep}{0.2\baselineskip}
\item $\Tr \left(A+B\right)=\Tr\left( A\right)+\Tr\left( B\right)$
\item $\Tr\left( \alpha A\right) =\alpha \Tr\left( A\right)$
\item $\begin{aligned}[t]
\Tr\left(AB\right) &= \Tr\left( BA\right) \longrightarrow & \Tr\left( ABC\right) &=\Tr\left( CAB\right) =\Tr\left( BCA\right)
\end{aligned}$
\item $g_{\mu\nu}g^{\mu\nu}=4$
\item $\begin{aligned}[t]
\gamma ^{\mu }\gamma ^{\nu }+\gamma ^{\nu }\gamma ^{\mu } &= 2g^{\mu \nu } \longrightarrow & \slashed{a}\slashed{b}+\slashed{b}\slashed{a}&=2a\cdot b
\end{aligned}$
\item $\gamma ^{\mu }\gamma ^{\nu }=4$
\item $\begin{aligned}[t]
\gamma _{\mu }\gamma ^{\nu }\gamma ^{\mu } &= -2\gamma ^{\nu } \longrightarrow & \gamma _{\mu }\slashed{a}\gamma^{\mu } &= -2\slashed{a}
\end{aligned}$
\item $\begin{aligned}[t] 
\gamma _{\mu }\gamma ^{\nu }\gamma ^{\lambda }\gamma ^{\mu }&= 4g^{\nu\lambda } \longrightarrow & \gamma _{\mu }\slashed{a}\slashed{b}\gamma ^{\mu} &= 4a\cdot b \end{aligned}$
\item $\begin{aligned}[t] 
\gamma _{\mu }\gamma ^{\nu }\gamma ^{\lambda }\gamma ^{\sigma}\gamma ^{\mu }&=-2\gamma^{\sigma}\gamma ^{\lambda}\gamma ^{\nu} \longrightarrow & \gamma _{\mu }\slashed{a}\slashed{b}\slashed{c}\gamma ^{\mu} &= -2\slashed{c}\slashed{b}\slashed{a}\end{aligned}$
\item Si se multiplica un número impar de matrices $\gamma^{\mu}$, la traza de la matriz resultante es siempre 0. $ \longrightarrow \Tr\left( \gamma ^{5}\gamma ^{\mu }\right) =\Tr\left( \gamma ^{5}\gamma ^{\mu }\gamma ^{\nu }\gamma ^{\lambda }\right) =0$
\item $\Tr\left(\mathbb{1}\right)=4$
\item $\begin{aligned}[t]
\Tr\left( \gamma ^{\mu }\gamma ^{\nu }\right)&=4g^{\mu \nu }\rightarrow & Tr\left(\slashed{a}\slashed{b}\right) &=4a\cdot b
\end{aligned}$
\item $\begin{aligned}[t]
\Tr\left( \gamma ^{\mu }\gamma ^{\nu }\gamma ^{\lambda }\gamma ^{\sigma }\right)&= 4\left( g^{\mu\nu}g^{\lambda \sigma} -g^{\mu \lambda }g^{\nu \sigma }+g^{\mu \sigma }g^{\nu \lambda }\right) \rightarrow \\ \Tr\left(\slashed{a}\slashed{b}\slashed{c}\slashed{d}\right) &=4\left( a\cdot bc\cdot d-a\cdot cb\cdot d+a\cdot db\cdot c\right)
\end{aligned}$
\item $\Tr\left( \gamma ^{5}\right)=0$
\item $\begin{aligned}[t]
\Tr\left( \gamma ^{5}\gamma ^{\mu }\gamma ^{\nu }\right)&=0 \rightarrow & \Tr\left(\gamma ^{5}\slashed{a}\slashed{b}\right) &=0 
\end{aligned}$
\item $\begin{aligned}[t]
\Tr\left( \gamma ^{5}\gamma ^{\mu }\gamma ^{\nu }\gamma ^{\lambda }\gamma ^{\sigma }\right)&=4i\varepsilon ^{\mu \nu \lambda \sigma } \rightarrow & \Tr\left(\gamma ^{5}\slashed{a}\slashed{b}\slashed{c}\slashed{d}\right) &=4i\varepsilon ^{\mu \nu \lambda \sigma }a_{\mu }b_{\nu }c_{\lambda }d_{\sigma } 
\end{aligned}$
\end{enumerate}

con $\varepsilon^{\mu\nu\lambda\sigma} = \left\{\begin{array}{lr}
        -1, & \text{si $\mu\nu\lambda\sigma$ es una permutación par de 0123},\\
        +1, & \text{si $\mu\nu\lambda\sigma$ es una permutación impar de 0123},\\
        0, & \text{si hay dos índices iguales}.
        \end{array}\right. $

\section{Helicidad y Quiralidad}\label{sec:quirality}
En las expresiones anteriores de la corriente débil aparece el operador $\gamma^5$, que indica la quiralidad de las partículas. A menudo, quiralidad y helicidad son conceptos confundidos. Por este motivo, es conveniente hacer un paréntesis y explicar con detalle en qué consiste cada uno.

La helicidad se define como la proyección del espín de una partícula en la dirección de su momento $\vec{p}$. En el marco teórico de la TCC, la helicidad de la partícula se representa mediante el operador helicidad $\mathcal{H}$:

\begin{equation}
\mathcal{H}=\dfrac{1}{2} \dfrac{\vec{p} \cdot \vec{\sigma}}{p}
\end{equation} 

Para una partícula fermiónica (de espín $1/2$), los autovalores de $\mathcal{H}$ son $+1/2$ (helicidad positiva), si la dirección de su espín coincide con la dirección de su movimiento, y $-1/2$ (helicidad negativa) en el caso contrario \cite{Bettini}. Así pues, los autoestados de helicidad se corresponden a espinores de dos componentes. En general, si la partícula tiene una masa distinta de cero, la helicidad no es un invariante de Lorentz.

La quiralidad es una propiedad de los bi-espinores que se utilizan para definir partículas y se representa por los autoestados del operador $\gamma^5$, con valores propios $\pm 1$. Los estados de quiralidad son designados como positivo o derecha (R) y negativo o izquierda (L), y sus proyectores son:

\begin{align}
\psi_L &= \dfrac{1-\gamma^5}{2}\psi & \psi_R &= \dfrac{1+\gamma^5}{2}\psi
\end{align}

A diferencia de la helicidad, la quiralidad es siempre un invariante de Lorentz. Sin embargo, cuando la partícula fermiónica carece de masa, helicidad y quiralidad coinciden.
El factor $\dfrac{1-\gamma^5}{2}$ es responsable de que la interacción débil presente estructura V-A, provocando a su vez que se viole la conservación de P y C en esta interacción. Todo esto implica que sólo los leptones con quiralidad negativa (izquierda) pueden reaccionar a la fuerza débil.