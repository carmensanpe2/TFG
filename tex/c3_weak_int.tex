\chapter{Interacción Débil}\label{cap:weak_int}

El origen de cada interacción fundamental se debe a causas diferentes. Por un lado, la existencia de carga eléctrica produce fuerzas electromagnéticas en las partículas y las hace interactuar entre sí, mientras que las interacciones fuertes se deben a la propiedad del color, mencionada en el capítulo anterior. No todas las partículas tienen carga ni color simultáneamente por lo que no todas son susceptibles a las mismas interacciones. El caso de la interacción débil es bastante interesante porque muchas partículas con propiedades distintas son sensibles a ella. Por ejemplo, los leptones no tienen carga de color, no ``sienten'' la interacción fuerte; pero los neutrinos no tienen carga eléctrica por lo que no pueden interactuar mediante fuerzas electromagnéticas. Sin embargo, ambos tipos de partículas pueden estar presentes en interacciones débiles.\cite{Griffiths2008} Además, la causa de la interacción débil, aunque no tiene un nombre específico, comúnmente se denota como \textit{carga débil}.

Los mesones $K$ y muchas otras partículas decaen por interacción débil. Desde el principio se consideró un tratamiento cuántico-relativista para su descripción y muchos científicos han contribuido al desarrollo de su formalismo. Cada interacción ocurre gracias al intercambio de una partícula mediadora o portadora de la fuerza de interacción. En la interacción fuerte es el gluón y en la electromagnética es el fotón. Sin embargo, en lo que respecta a la interacción débil, esas partículas mediadoras encargadas de transmitir la fuerza débil entre los quarks y leptones, son los bosones vectoriales, llamados así porque tienen espín 1. Debido a su gran masa, se tiene en consecuencia que la interacción débil es de muy corto alcance. Estos bosones portadores de la fuerza débil pueden ser cargados $W^{\pm}$ o neutros $Z^0$, dependiendo de las partículas que participan en el proceso. 

Hasta la década de los 70, sólo se habían observado procesos de intercambio de bosones cargados $W^{\pm}$. En los años 60 se empezó a formular una teoría que aunaba la interacción débil junto con la electromagnética, en lo que se conoce hoy en día como \textit{Teoría Electrodébil}. Esta teoría predecía la existencia del bosón neutro mediador de la fuerza débil y dicha hipótesis fue confirmada experimentalmente en 1973 \cite{BrianM}, con el descubrimiento de $Z^0$.

Este capítulo se centra, principalmente, en la descripción de procesos de interacción débil cargados con intercambio de bosones $W^{\pm}$ y nos servirá como base para el estudio del decaimiento de mesones $K$ cargados.

\section{Formalismo de la Interacción Débil}\label{cap:formalism}
\subsection{Los campos y su cuantización}\label{cap:quantization}
De acuerdo con la Teoría Cuántica de Campos (TCC), cada partícula lleva asociado un campo $\phi(x)$, $\psi(x)$ o $W_{\mu}(x)$ que son funciones que dependen del espacio y del tiempo $x^{\mu}=(ct,\boldsymbol{\vec{x}})$. 
\subsection{Teoría de la Perturbación}\label{cap:perturbation_theory}

\section{Decaimiento de mesones $K$ cargados}
\label{charged_kaon_decay}
\vspace{5mm}