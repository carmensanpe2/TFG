\chapter{Interacción Débil}\label{cap:weak_int}
El origen de cada interacción fundamental se debe a causas diferentes. Por un lado, la existencia de carga eléctrica produce fuerzas electromagnéticas en las partículas y las hace interactuar entre sí, mientras que la interacción fuerte se debe a la propiedad del color, mencionada en el capítulo anterior. No todas las partículas tienen carga ni color simultáneamente por lo que no todas son susceptibles a las mismas interacciones. El caso de la interacción débil es bastante interesante porque muchas partículas con propiedades distintas son sensibles a ella. Por ejemplo, los leptones no tienen carga de color, no ``sienten'' la interacción fuerte; pero los neutrinos no tienen carga eléctrica por lo que no pueden interactuar mediante fuerzas electromagnéticas. Sin embargo, ambos tipos de partículas pueden estar presentes en interacciones débiles.\cite{Griffiths2008} Además, la causa de la interacción débil, aunque no tiene un nombre específico, comúnmente se denota como \textit{carga débil}.

Los mesones $K$ y muchas otras partículas decaen por interacción débil. Desde el principio se consideró un tratamiento cuántico-relativista para su descripción y muchos científicos han contribuido al desarrollo de su formalismo. Cada interacción ocurre gracias al intercambio de una partícula mediadora o portadora de la fuerza de interacción. En la interacción fuerte es el gluón y en la electromagnética es el fotón. Sin embargo, en lo que respecta a la interacción débil, esas partículas mediadoras encargadas de transmitir la fuerza débil entre los quarks y leptones, son los bosones vectoriales, llamados así porque tienen espín 1. Debido a su gran masa, se tiene en consecuencia que la interacción débil es de muy corto alcance. Estos bosones portadores de la fuerza débil pueden estar cargados $W^{\pm}$ eléctricamente o ser neutros $Z^0$, según las partículas que participan en el proceso. 

Hasta la década de los 70, sólo se habían observado procesos de intercambio de bosones cargados $W^{\pm}$. En los años 60 se empezó a formular una teoría que aunaba la interacción débil junto con la electromagnética, conocida hoy en día como \textit{Teoría Electrodébil}. Esta teoría predecía la existencia del bosón neutro mediador de la fuerza débil y dicha hipótesis fue confirmada experimentalmente en 1973 \cite{BrianM}, con el descubrimiento de $Z^0$.

Este capítulo se centra, principalmente, en la descripción de procesos de interacción débil con intercambio de bosones $W^{\pm}$ o procesos de in\textit{teracción débil de corriente cargada} y nos servirá como base para el estudio del decaimiento de mesones $K$ cargados.

\section{Formalismo de la Interacción Débil}\label{cap:formalism}
\subsection{Interacciones en Teoría Cuántica de Campos}\label{sec:qft}
De acuerdo con la Teoría Cuántica de Campos (TCC), cada partícula lleva asociado un campo cuántico $\phi(x)$, $\psi(x)$ o $W_{\mu}(x)$, dependiente del espacio y del tiempo $x^{\mu}=(ct,\boldsymbol{\vec{x}})$\protect\footnotemark. Estos campos $\phi(x)$, $\psi(x)$ o $W_{\mu}(x)$ actúan como operadores encargados de aniquilar partículas o crear antipartículas de espín $0$, $1/2 $ y $1$, respectivamente \cite{notas2020}.

\footnotetext{Consultar el Apéndice \hyperref[cap:A]{A} para más información sobre la notación utilizada en este capítulo.}
 
Para describir la evolución espacio-temporal de dichos campos asociados a partículas se hace uso de la densidad lagrangiana $\mathcal{L}$. Para las distintas partículas y sin tener en cuenta la interacción, en el sistema natural de unidades ($c=\hbar =1$), $\mathcal{L}$ tiene las siguientes expresiones:
\begin{itemize}
\item Bosones de espín $0$, tales como los mesones $K$ o los piones:
\end{itemize}
\begin{equation*}
\mathcal{L}=-\partial ^{\mu }\phi \left( x\right) ^{\ast }\partial_{\mu} \phi \left( x\right) -m^{2}\phi \left( x\right) ^{\ast }\phi \left( x\right)
\end{equation*}
El campo $\phi(x)$ tiene carácter escalar y es responsable de (crear) aniquilar (anti)partículas con $J=0$. $\phi^{\ast}(x)$ es su campo conjugado y hace justo lo opuesto: crea partículas y aniquila antipartículas con $J=0$.
\begin{itemize}
\item Fermiones de espín $1/2$, como los leptones:
\end{itemize}
\begin{equation*}
\mathcal{L}=-\overline{\psi }\left( x\right) \left( \slashed{\partial}+m\right) \psi \left( x\right)
\end{equation*}
$\psi(x)$ tiene carácter de espinor y (crea) aniquila (anti)partículas con $J=1/2$. Su adjunto también hace lo contrario.
\begin{itemize}
\item Bosones de espín $1$, como son los fotones o los bosones vectoriales $W$:
\end{itemize}
\begin{equation*}
\mathcal{L}=-\dfrac{1}{4}\left| \partial _{\mu }W^{\nu }\left( x\right) -\partial _{\nu }W^{\mu }\left( x\right) \right| ^{2}-\dfrac{1}{2}m^{2}\left| W^{\mu }\left( x\right) \right| ^{2}
\end{equation*}
Por último, $W^{\mu}(x)$ posee carácter de cuadrivector y se encarga de crear antipartículas y aniquilar partículas con $J=1$, mientras que su conjugado hace, nuevamente, lo opuesto.

Del mismo modo, las interacciones se describen mediante unas constantes, denominadas \textit{contantes de acoplamiento}, y productos entre los campos cuánticos de las partículas que intervienen en el proceso. Como $\mathcal{L}$ debe ser un escalar, sólo se permiten combinaciones entre campos que resulten en invariantes de Lorentz \cite{notas2020}. Esto es posible, ya que la TCC, entiende las interacciones como un intercambio de partículas mediadoras, tal y como se mencionó anteriormente. En su libro \cite{Bettini}, Bettini lo explica con el siguiente ejemplo: se tiene una partícula $a$ que interactúa en el campo mediado por el bosón $V$; en el vacío, $a$ está continuamente emitiendo y absorbiendo este bosón, tal y como se muestra en \ref{fig:bettini1}. No obstante, si una partícula $b$ se encuentra cerca de $a$ y tiene su misma interacción, puede absorber un bosón $V$ que previamente haya sido emitido por $a$ (ver \ref{fig:bettini2}). Entonces, se puede afirmar que $a$ y $b$ interactúan entre sí intercambiando un bosón $V$, es decir, combinando sus campos cuánticos.
\begin{figure}[h]
\begin{subfigure}{.5\textwidth}
  \centering
\includegraphics[width=0.6\linewidth]{{C:/Users/Carmen/Desktop/Universidad/TFG/Borradores/img/bettini1.PNG}}
\caption{$V$ emitido y reabsorbido por $a$}
  \label{fig:bettini1}
\end{subfigure}%
\begin{subfigure}{.5\textwidth}
  \centering
  \includegraphics[width=0.6\linewidth]{{C:/Users/Carmen/Desktop/Universidad/TFG/Borradores/img/bettini2.PNG}}
  \caption{$V$ emitido por $a$ y absorbido por $b$}
  \label{fig:bettini2}
\end{subfigure}
\caption[Esquematización del proceso de interacción en TCC]{Proceso de interacción mediante intercambio del bosón mediador.  \cite{Bettini}}
\label{fig:bettini}
\end{figure}

Continuando con el ejemplo anterior, Bettini indica que el bosón mediador $V$ tiene, en general, una masa $m$ no nula, lo que provoca que, durante su emisión, se viole momentáneamente la conservación de energía $\Delta E = m$. Lo mismo, pero de forma opuesta ocurre durante su absorción. Así pues, la violación neta dura un $\Delta t$ y satisface \textit{El Principio de Incertidumbre de Heisenberg}: $\Delta E \Delta t \leq \hbar$, lo que implica que $V$ sólo puede alejarse una distancia finita $R=c\Delta t$. Esta distancia equivale al rango de la fuerza de interacción, por lo tanto, cuanta mayor masa tenga el bosón mediador de una interacción, menor será su rango de alcance \cite{Bettini}. Dado que los bosones mediadores $W^{\pm}$ y $Z^0$ tienen una masa muy grande ($m_W=80,379 \pm 0,012$ GeV y $m_Z=91,1876 \pm 0,0021$ GeV, respectivamente \cite{Zyla}), la interacción débil tiene un alcance muy corto, más que cualquier otra interacción fundamental; de ahí la denominación de ``débil''. 

Gráficamente, las interacciones se representan con Diagramas de Feynmann. Estos diagramas proporcionan información acerca de la amplitud de probabilidad de los distintos procesos, donde cada línea representa a una partícula y cada vértice corresponde a cada actuación del lagrangiano de la interacción $\mathcal{L}$. Las líneas externas corresponden a partículas reales, mientras que las internas, que conectan los vértices, representan partículas virtuales, conocidas como propagadores o mediadores. El propagador es la partícula que se crea y aniquila; la que media la interacción. \cite{notas2020} En nuestro ejemplo anterior, dicho propagador sería el bosón $V$  y en el caso de la interacción débil, los propagadores serían los bosones vectoriales $W^{\pm}$ o $Z^0$.

La densidad lagrangiana $\mathcal{L}$ que describe cada vértice en el Diagrama de Feynmann de una Interacción de corriente cargada débil, tiene esta forma:
\begin{equation}
\mathcal{L}^{w}=\dfrac{g_{w}}{\sqrt{2}}\left( W^{\mu }\left( x\right) j_{\mu}^{+}\left( x\right) +\left[ W^{\mu }\left( x\right) \right]^{\ast }j_{\mu}^{-}\left( x\right) \right)\label{eq:weak_lagrangian}
\end{equation}
Recordemos que el campo $W^{\mu}(x)$ aniquila $W^{+}$ o crea $W^{-}$ mientras que su conjugado $\left(W^{\mu}(x)\right)^\ast$ aniquila $W^{-}$ o crea $W^{+}$. En un proceso que ocurre por interacción débil con intercambio de $W^{\pm}$, la carga neta del estado inicial y el final difieren en una unidad y, entonces, se habla de interacción de corriente cargada. Luego, la densidad de corriente débil $j_{\mu}$ puede ser positiva o negativa, y se compone de dos términos, uno para la corriente leptónica y otro para la hadrónica:
\begin{equation}
j_{\mu} ^{\pm }\left( x\right) =j_{\mu} ^{\pm lep}\left( x\right) +j_{\mu} ^{\pm had}\left( x\right) \label{eq:weak_current_hadylep}
\end{equation}

La corriente leptónica es una composición de corrientes de cada familia de leptones, así se tiene un término para los electrones, otro para los muones y otro para los taones:
\begin{equation}
j_{\mu }^{\pm lep}\left( x\right) =j_{\mu }^{\pm el}\left( x\right) +j_{\mu }^{\pm muon}\left( x\right) +j_{\mu} ^{\pm tau}\left( x\right)\label{eq:leptonic_weak_current}
\end{equation}
La corriente leptónica de electrones puede expresarse de la siguiente forma:
\begin{align}
j_{\mu }^{-el}\left(x\right)&=i\overline{\psi}_{e}\left( x\right) \dfrac{1-\gamma_{5}}{2}\gamma _{\mu }\psi_{{ \nu}_{e}}\left( x\right) & j_{\mu}^{+el}\left(x\right)&= i\overline{\psi}_{{\nu}_{e}}\left(x\right) \dfrac{1-\gamma_{5}}{2}\gamma _{\mu}\psi_{e}\left( x\right)\label{eq:electric_weak_current}
\end{align}
La corriente negativa $j_{\mu }^{-el}$, aniquila $\nu_e$ (o crea $\overline{\nu_e}$) y crea $e^-$ (o aniquila $e^+$), mientras que la corriente positiva $j_{\mu }^{+el}$ hace justo lo opuesto. $j_{\mu }^{\pm muon}$ y $j_{\mu }^{\pm tau}$ pueden definirse de manera análoga. Estas corrientes leptónicas se caracterizan por conservar el número cuántico leptónico y la carga de las partículas que intervienen.

El operador $\gamma^5$ tiene dos autoestados $\pm 1$, asociados cada uno a la quiralidad positiva o negativa, respectivamente. Cuando las partículas fermiónicas tienen una velocidad próxima a la de la luz, la quiralidad se relaciona con su helicidad. El factor $\dfrac{1-\gamma^5}{2}$ es responsable de que sólo los leptones con quiralidad negativa reaccionen a la interacción débil y explica por qué, en ocasiones, hay una violación de paridad P y CP en la interacción débil, tal y como se detallará en el capítulo siguiente.

El operador de corriente hadrónica $j_{\mu} ^{\pm had}$ es el encargado de crear o aniquilar hadrones, conservando siempre el número bariónico $B$ e incrementando o reduciendo en una unidad la carga eléctrica total $Q$. Sin embargo, aunque no siempre, también son capaces de modificar la extrañeza: la corriente hadrónica positiva (negativa) puede aumentar (disminuir) la extrañeza en una unidad $\Delta S = +1$ ($\Delta S = -1$) \cite{notas2020}.

Además, $g_W$ es la constante de acoplo de la interacción débil y se relaciona con la constante de Fermi $G_F$ según \ref{eq:fermi_coupling}. Experimentalmente se ha comprobado que $G_F$ tiene un valor único para todos los procesos donde interviene la interacción débil, siendo este $G_{F}=1,166 \cdot 10^{-5}$ $GeV^{−2}$.
\begin{equation}
\dfrac{G_{F}}{g_{w}}=\dfrac{\sqrt{2}}{8m_{w}}\label{eq:fermi_coupling}
\end{equation}

\subsection{Interacción débil en el Modelo de Quarks}\label{sec:weak_int_quarks}
En el modelo de Quarks, la interacción débil se describe también mediante Diagramas de Feynmann pero esta vez se representa por medio de un cambio de sabor en los quarks, emitiendo a su vez bosones $W^{\pm}$. Según si un proceso de interacción débil, modifica o no su extrañeza, la expresión de corriente hadrónica puede variar:
\begin{itemize}
\item Si no cambia la extrañeza: se aniquila un quark $d$ y se crea un quark $u$.
\end{itemize}
\begin{equation*}
j_{\mu}^{+had}(\Delta S= 0)=\cos \left( \theta _{c}\right) i\overline{\psi }_{u}\left( x\right) \dfrac{1-\gamma _{5}}{2}\gamma _{\mu }\psi _{d}
\end{equation*}
\begin{itemize}
\item Si se modifica la extrañeza: se aniquila un quark $s$ y se crea un quark $u$.
\end{itemize}
\begin{equation*}
j_{\mu}^{+had}(\Delta S= 0)=\sin \left( \theta _{c}\right) i\overline{\psi }_{u}\left( x\right) \dfrac{1-\gamma _{5}}{2}\gamma _{\mu }\psi _{s}
\end{equation*}

En la interacción débil, el sabor $u$, de carga $+2/3$, se conecta con los otros dos sabores ligeros mediante $d\cdot \cos \left( \theta _{c}\right) +s\cdot \sin \left( \theta _{c}\right)$, donde $\theta _{c}$ hace referencia al ángulo de Cabbibo, cuyo valor experimental ha resultado ser $13,02\degree$. Del mismo modo, Cabbibo planteó que debería haber otro quark con la misma carga que $u$ pero que se conectara con la combinación ortogonal $-d\cdot \sin \left( \theta _{c}\right) +s\cdot \cos \left( \theta c\right)$, prediciendo así la existencia del quark pesado $c$.

Posteriormente, se concluyó que, en la interacción débil, todos los sabores de quarks, pesados y ligeros, debían estar conectados mediante la \textit{matriz CKM} (Cabibbo-Kobayashi-Maskawa). La importancia de esta matriz recae en sus términos diagonales, conectando: $u\leftrightarrow d$, $c\leftrightarrow s$ y $t\leftrightarrow b$. Los términos no diagonales son los responsables de que los quarks pesados vayan decayendo progresivamente en los sabores ligeros $u$ y $d$, que son los constituyentes predominantes de la materia ordinaria. La expresión de la matriz CKM es:
\begin{align}
d' &= V_{ud}\cdot d+V_{us}\cdot s + V_{ub}\cdot b \nonumber \\
s' &= V_{cd}\cdot d+V_{cs}\cdot s + V_{cb}\cdot b \\
b' &= V_{td}\cdot d+V_{ts}\cdot s + V_{tb}\cdot b \nonumber\label{eq:CKM_matrix}
\end{align}

\subsection{Teoría Electrodébil}\label{sec:electroweak}
Describir unificación de la interacción débil y la electromagnética

\section{Decaimiento de mesones $K$ cargados}
\label{sec:charged_kaon_decay}
Los mesones $K^{\pm}$ decaen por interacción débil y sus procesos de decaimiento (modos) pueden clasificarse en varias categorías. A continuación presentamos los modos leptónicos, semileptónicos y hadrónicos, que son los más relevantes:

\begin{table}[!htb]
\begin{minipage}{.5\linewidth}
    \centering
\begin{tabular}{ c c } 
\toprule
\makecell{Mesón $K^{+}$}  &  Mesón $K^{-}$ \\
\midrule   
$e^{+}\nu_{e}$ & $e^{-} \overline{\nu_{e}}$ \\
$\mu^{+}\nu_{\mu}$ & $\mu^{-} \overline{\nu_{e}}$ \\
$\pi^{0} e^{+} \nu_{e}$ & $\pi^{0} e^{-} \overline{\nu_{e}}$ \\
$\pi^{0} \mu^{+}\nu_{\mu}$ & $\pi^{0} \mu^{-} \overline{\nu_{e}}$ \\
$\pi^{0}\pi^{0} e^{+}\nu_{e}$ & $\pi^{0}\pi^{0} e^{-} \overline{\nu_{e}}$ \\
$\pi^{+}\pi^{-} e^{+}\nu_{e}$ & $\pi^{+}\pi^{-} e^{-} \overline{\nu_{e}}$ \\
$\pi^{+}\pi^{-} \mu^{+}\nu_{\mu}$ & $\pi^{+}\pi^{-} \mu^{-} \overline{\nu_{e}}$ \\
$\pi^{0}\pi^{0}\pi^{0} e^{+}\nu_{e}$ & $\pi^{0}\pi^{0}\pi^{0} e^{-} \overline{\nu_{e}}$ \\
\bottomrule
\end{tabular}
\caption[Modos de decaimiento leptónicos y semileptónicos de $K^{\pm}$]{Modos (semi-)leptónicos. \cite{Zyla}}
\label{tab:Kpm_leptonic_decay}
\end{minipage}\hfill
\begin{minipage}{.5\linewidth}
    \centering
\begin{tabular}{ c c } 
    \toprule
    \makecell{Mesón $K^{+}$}  &  Mesón $K^{-}$ \\    
    \midrule
$\pi^{+}\pi^{0}$ & $\pi^{-}\pi^{0}$ \\
$\pi^{+}\pi^{0}\pi^{0}$ & $\pi^{-}\pi^{0}\pi^{0}$ \\
$\pi^{+}\pi^{+}\pi^{-}$ & $\pi^{+}\pi^{-}\pi^{-}$ \\
    \bottomrule
\end{tabular}
\caption[Modos de decaimiento hadrónicos de $K^{\pm}$]{Modos hadrónicos. \cite{Zyla}}
\label{tab:Kpm_hadronic_decay}
\end{minipage}
\end{table}

Como puede observarse, los modos de decaimiento de $K^{-}$ son los mismos modos que los de $K^{+}$ pero con carga conjugada. Sin embargo, no todos estos procesos tienen la misma probabilidad de ocurrir. ¿A qué se debe esto y cuáles son los factores que determinan esta probabilidad?. Para responder a estas preguntas nos centramos en el estudio de los modos leptónicos del kaón positivo: $K^{+} \rightarrow e^{+}\nu_{e}$ y $K^{+} \rightarrow \mu^{+}\nu_{e}$.


\subsection{Teoría de la Perturbación}\label{sec:perturbation_theory}
Para calcular la probabilidad de decaimiento de los mesones $K$ hacemos uso de la siguiente expresión:

\begin{equation}
W_{if}= \dfrac{2\pi }{\hbar }\left| \langle i\right| \widehat{H'}| f\rangle| ^{2}\rho \left( E\right)
\end{equation}