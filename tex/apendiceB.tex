\appendix
\chapter*{Apéndice B}\label{cap:B}
\addcontentsline{toc}{chapter}{Apéndice B}
\setcounter{section}{0}
\renewcommand{\thesection}{B.\arabic{section}}
\setcounter{table}{0}
\renewcommand{\thetable}{B.\arabic{table}}
\setcounter{equation}{0}
\renewcommand{\theequation}{B.\arabic{equation}}

\section{Cálculos detallados del modo leptónico de $\PKm$}\label{sec:detailed_calc}

El decaimiento leptónico $\PKm \rightarrow \Plm + \Pagnl$ puede observarse en la figura \ref{fig:diagrama1}. La asignación de cuadrivectores momento a cada partícula se muestra en la figura \ref{fig:diagrama2}:
\begin{itemize}
\setlength{\itemsep}{0.2\baselineskip}
\item mesón inicial $\PKm$: masa $m_K$ y cuadri-momento $p$.
\item bosón intercambiado $\PWm$: masa $M_W$ y cuadri-momento $q$.
\item antineutrino final $\Pagnl$: masa $m_{\nu}=0$ y cuadri-momento $p_2$.
\item leptón final $\Plm$: masa $m_{\Pl}$ y cuadri-momento $p_3$.
\end{itemize}
A partir del razonamiento expuesto en la sección \ref{sec:charged_kaon_decay} y empleando unidades naturales, se llega a la expresión de la integral:
\begin{multline}
\int[\overline{u}\left( 3\right) \left( \dfrac{-ig_w}{2\sqrt{2}}\right) ^{2}\gamma ^{\mu }\left( 1-\gamma ^{5}\right) v\left( 2\right)]F^{\mu} \dfrac{ig{_{\mu \nu}}}{\left( M_W\right)^{2}}\\ \times \left( 2\pi \right) ^{4}\delta ^{4}\left( q-p_{2}-p_{3}\right) \cancel{\left( 2\pi \right)^{4}}\delta ^{4}\left( p-q\right) \dfrac{d^{4}q}{\cancel{\left( 2\pi \right) ^{4}}}
\end{multline}
Integramos sabiendo que $p=q$, obteniendo lo siguiente:
\begin{equation}
[\overline{u}\left( 3\right) \left( \dfrac{-ig_w}{2\sqrt{2}}\right) ^{2}\gamma ^{\mu }\left( 1-\gamma ^{5}\right) v\left( 2\right)]F^{\mu} \dfrac{ig{_{\mu \nu}}}{\left( M_W\right)^{2}}\left( 2\pi \right) ^{4}\delta ^{4}\left(p-p_{2}-p_{3}\right)
\end{equation}
A continuación, nos deshacemos del término $\left( 2\pi \right) ^{4}\delta ^{4}\left(p-p_{2}-p_{3}\right)$ e igualamos a $-i\mathcal{M}$, de acuerdo a las reglas de Feynman para calcular la amplitud de probabilidad $\mathcal{M}$:
\begin{equation}
[\overline{u}\left( 3\right) \left( \dfrac{-ig_w}{2\sqrt{2}}\right) ^{2}\gamma ^{\mu }\left( 1-\gamma ^{5}\right) v\left( 2\right)]F^{\mu} \dfrac{ig{_{\mu \nu}}}{\left( M_W\right)^{2}}\cancel{\left( 2\pi \right) ^{4}\delta ^{4}\left(p-p_{2}-p_{3}\right)}=-i\mathcal{M}
\end{equation}
Finalmente, se obtiene:
\begin{equation}
\mathcal{M} =\dfrac{{g_{w}}^2}{8\left( M_W\right)^{2}}\left[ \overline{u}\left(3\right) \gamma^{\mu}\left( 1-\gamma ^{5} \right) v\left( 2\right) \right] F^{\mu}
\end{equation}
con $F^{\mu}=f_K p^{\mu}$.

Sumando sobre todos los espines posibles de las partículas finales y haciendo la media, se llega a esta expresión:
\begin{equation}
\left\langle |\mathcal{M}|^{2}\right\rangle=\left[ \dfrac{f_{K}}{8}\left( \dfrac{g_w}{M_W}\right)^{2}\right]^{2} p_{\mu }p_{\nu} \left[ \overline{u}\left( 3\right)\gamma^{\mu}\left( 1-\gamma^{5}\right) v\left( 2\right) \right] \left[ \overline{u}\left( 3\right) \gamma^{\nu} \left( 1-\gamma^{5}\right) v\left( 2\right) \right] ^{\ast }
\end{equation}

Aplicando el truco de Casimir \ref{eq:casimir_trick}, nos deshacemos de los espinores y dejamos la expresión en función de las trazas:
\begin{multline}
\left[ \overline{u}\left( 3\right)\gamma^{\mu}\left( 1-\gamma^{5}\right) v\left( 2\right) \right] \left[ \overline{u}\left( 3\right) \gamma^{\nu }\left( 1-\gamma^{5}\right) v\left( 2\right) \right] \\ \equiv \Tr \left[ \gamma^{\mu } \left(1 -\gamma ^{5}\right) \slashed{p_{2}}\gamma^{\nu }\left( 1-\gamma ^{5}\right) \left( \slashed{p_{3}}+m_{\Pl}\right) \right]
\end{multline}

Expandiendo este producto de trazas:
\begin{multline}
\Tr \left[ \gamma^{\mu } \left(1 -\gamma ^{5}\right) \slashed{p_{2}}\gamma^{\nu }\left( 1-\gamma ^{5}\right) \left( \slashed{p_{3}}+m_{\Pl}\right) \right] = \Tr[ \gamma ^{\mu }\slashed{p_{2}}\gamma ^{\nu }\slashed{p_{3}} + \cancel{\gamma ^{\mu }\slashed{p_{2}}\gamma ^{\nu }m_{\Pl}} -\gamma ^{\mu }\slashed{p_{2}}\gamma ^{\nu }\gamma ^{5}\slashed{p_{3}} \\ - \cancel{\gamma ^{\mu }\slashed{p_{2}}\gamma ^{\nu }\gamma ^{5}m_{\Pl}} -\gamma ^{\mu }\gamma ^{5}\slashed{p_{2}}\gamma ^{\nu }\slashed{p_{3}}- \cancel{\gamma ^{\mu }\gamma ^{5}\slashed{p_{2}}\gamma ^{\nu }m_{\Pl}} +\gamma ^{\mu }\gamma ^{5}\slashed{p_{2}}\gamma ^{\nu }\gamma ^{5}\slashed{p_{3}} + \cancel{\gamma ^{\mu }\gamma ^{5}\slashed{p_{2}}\gamma ^{\nu }\gamma ^{5}m_{\Pl}}
\end{multline}

Los términos anteriores se han tachado por la propiedad 10 de las trazas (ver Apéndice \hyperref[cap:A]{A}). Usando la propiedad 3, reescribimos como:
\begin{multline}
\Tr[ \gamma ^{\mu }\slashed{p_{2}}\gamma ^{\nu }\slashed{p_{3}} -\gamma ^{\mu }\slashed{p_{2}}\gamma ^{\nu }\gamma ^{5}\slashed{p_{3}} -\gamma ^{\mu }\gamma ^{5}\slashed{p_{2}}\gamma ^{\nu }\slashed{p_{3}} +\gamma ^{\mu }\gamma ^{5}\slashed{p_{2}}\gamma ^{\nu }\gamma ^{5}\slashed{p_{3}}] = \\ \Tr[ \gamma ^{\mu }\slashed{p_{2}}\gamma ^{\nu }\slashed{p_{3}}] - \Tr[\gamma ^{\mu }\slashed{p_{2}}\gamma ^{\nu }\gamma ^{5}\slashed{p_{3}}] - \Tr[\gamma ^{\mu }\gamma ^{5}\slashed{p_{2}}\gamma ^{\nu }\slashed{p_{3}}] + \Tr[\gamma ^{\mu }\gamma ^{5}\slashed{p_{2}}\gamma ^{\nu }\gamma ^{5}\slashed{p_{3}}]
\end{multline}

Evaluando cada traza recordando la notación de Feynmann (eq. \ref{eq:slashnot}) y las propiedades de la sección \ref{sec:trazas}:

\begin{itemize}
\item $\begin{multlined}[t]
\Tr[ \gamma ^{\mu }\slashed{p_{2}}\gamma ^{\nu }\slashed{p_{3}}]= \left( p_{2}\right) _{\lambda }\left( p_{3}\right) _{\sigma } \Tr\left[ \gamma ^{\mu }\gamma ^{\lambda }\gamma ^{\nu }\gamma ^{\sigma }\right] = \\ \left( p_{2}\right) _{\lambda }\left( p_{3}\right) _{\sigma }4\left( g^{\mu\nu}g^{\lambda \sigma} -g^{\mu \lambda }g^{\nu \sigma }+g^{\mu \sigma }g^{\nu \lambda }\right) = 4\left[{p_{2}}^{\mu }{ p_{3}}^{\nu }-g_{\mu \nu }\left( p_{2}\cdot p_{3}\right) +{p_{3}}^{\mu } {p_{2}}^{\nu }\right]
\end{multlined}$
\item $\begin{multlined}[t]
- \Tr[\gamma ^{\mu }\slashed{p_{2}}\gamma ^{\nu }\gamma ^{5}\slashed{p_{3}}] = -\left( p_{2}\right) _{\lambda }\left( p_{3}\right) _{\sigma } \Tr\left[ \gamma ^{\mu }\gamma ^{\lambda }\gamma ^{\nu }\gamma^5 \gamma ^{\sigma }\right] = \\ -\left( p_{2}\right) _{\lambda }\left( p_{3}\right) _{\sigma } \Tr\left[\gamma^5 \gamma ^{\sigma }\gamma ^{\mu }\gamma ^{\lambda }\gamma ^{\nu }\right]=-\left( p_{2}\right) _{\lambda }\left( p_{3}\right) _{\sigma }4i\varepsilon ^{\sigma \mu \lambda \nu }=-\left( p_{2}\right) _{\lambda }\left( p_{3}\right) _{\sigma }4i\varepsilon ^{\mu \nu \lambda \sigma}
\end{multlined}$
\item $\begin{multlined}[t]
- \Tr[\gamma ^{\mu }\gamma ^{5}\slashed{p_{2}}\gamma ^{\nu }\slashed{p_{3}}]= -\left( p_{2}\right) _{\lambda }\left( p_{3}\right) _{\sigma } \Tr\left[ \gamma ^{\mu }\gamma^5 \gamma ^{\lambda }\gamma ^{\nu }\gamma ^{\sigma }\right] = \\ -\left( p_{2}\right) _{\lambda }\left( p_{3}\right) _{\sigma } \Tr\left[\gamma^5\gamma ^{\lambda }\gamma ^{\nu \gamma ^{\sigma }\gamma ^{\mu }}\right]= -\left( p_{2}\right) _{\lambda }\left( p_{3}\right) _{\sigma }4i\varepsilon ^{\lambda \nu \sigma \mu} = -\left( p_{2}\right) _{\lambda }\left( p_{3}\right) _{\sigma }4i\varepsilon ^{\mu \nu \lambda \sigma}
\end{multlined}$
\item $\begin{multlined}[t]
\Tr[ \gamma ^{\mu }\gamma ^{5}\slashed{p_{2}}\gamma ^{\nu }\gamma ^{5}\slashed{p_{3}}]= \left( p_{2}\right) _{\lambda }\left( p_{3}\right) _{\sigma } \Tr\left[ \gamma ^{\mu }\gamma ^{5}\gamma ^{\lambda }\gamma ^{\nu }\gamma ^{5}\gamma ^{\sigma }\right] = \left( p_{2}\right) _{\lambda }\left( p_{3}\right) _{\sigma } \Tr\left[\gamma ^{5}\gamma ^{5} \gamma ^{\mu }\gamma ^{\lambda }\gamma ^{\nu }\gamma ^{\sigma }\right]= \\ \left( p_{2}\right) _{\lambda }\left( p_{3}\right) _{\sigma }4\left( g^{\mu\nu}g^{\lambda \sigma} -g^{\mu \lambda }g^{\nu \sigma }+g^{\mu \sigma }g^{\nu \lambda }\right) = 4\left[{p_{2}}^{\mu }{ p_{3}}^{\nu }-g_{\mu \nu }\left( p_{2}\cdot p_{3}\right) +{p_{3}}^{\mu } {p_{2}}^{\nu }\right]
\end{multlined}$ 
\end{itemize}

Aunando los cuatro resultados anteriores de las trazas, se tiene que:
\begin{multline}
\Tr \left[ \gamma^{\mu } \left(1 -\gamma ^{5}\right) \slashed{p_{2}}\gamma^{\nu }\left( 1-\gamma ^{5}\right) \left( \slashed{p_{3}}+m_{l}\right) \right] = 4\left[{p_{2}}^{\mu }{ p_{3}}^{\nu }-g_{\mu \nu }\left( p_{2}\cdot p_{3}\right) +{p_{3}}^{\mu } {p_{2}}^{\nu }\right] \\ -4i\left( p_{2}\right) _{\lambda }\left( p_{3}\right) _{\sigma }\varepsilon ^{\mu \nu \lambda \sigma} -4i\left( p_{2}\right) _{\lambda }\left( p_{3}\right) _{\sigma }\varepsilon ^{\mu \nu \lambda \sigma} + 4\left[{p_{2}}^{\mu }{ p_{3}}^{\nu }-g_{\mu \nu }\left( p_{2}\cdot p_{3}\right) +{p_{3}}^{\mu } {p_{2}}^{\nu }\right] = \\ 8\left[{p_{2}}^{\mu }{ p_{3}}^{\nu }-g_{\mu \nu }\left( p_{2}\cdot p_{3}\right) +{p_{3}}^{\mu } {p_{2}}^{\nu } -i{p_{2}}^{\lambda} {{p_{3}}}^{\sigma} \varepsilon^{  \mu \nu \lambda \sigma} \right]
\end{multline}

Por lo tanto, queda:
\begin{equation}
\left\langle |\mathcal{M}|^{2}\right\rangle=\left[ \dfrac{f_{K}}{\cancel{8}}\left( \dfrac{g_w}{M_W}\right)^{2}\right]^{2} p_{\mu }p_{\nu}\cancel{8}\left[{p_{2}}^{\mu }{ p_{3}}^{\nu }-g_{\mu \nu }\left( p_{2}\cdot p_{3}\right) +{p_{3}}^{\mu } {p_{2}}^{\nu } -i{p_{2}}^{\lambda} {{p_{3}}}^{\sigma} \varepsilon^{  \mu \nu \lambda \sigma} \right]
\end{equation}

Resolviendo los productos de tensores:
\begin{multline}
p_{\mu }p_{\nu}\left[{p_{2}}^{\mu }{ p_{3}}^{\nu }-g_{\mu \nu }\left( p_{2}\cdot p_{3}\right) +{p_{3}}^{\mu } {p_{2}}^{\nu } -i{p_{2}}^{\lambda} {{p_{3}}}^{\sigma} \varepsilon^{  \mu \nu \lambda \sigma} \right] =  p_{\mu }p_{\nu}{p_{2}}^{\mu }{ p_{3}}^{\nu } + p_{\mu }p_{\nu}{p_{3}}^{\mu } {p_{2}}^{\nu } \\ - p_{\mu }p_{\nu}g_{\mu \nu }\left( p_{2}\cdot p_{3}\right)- \cancel{i p_{\mu }p_{\nu}{p_{2}}^{\lambda} {{p_{3}}}^{\sigma} \varepsilon^{  \mu \nu \lambda \sigma}} = 2\left( p\cdot p_{2}\right) \left( p\cdot p_{3}\right) -p^{2}\left( p_{2}\cdot p_{3}\right)
\end{multline}
El término tachado es nulo por la propiedad 16.

Finalmente, se tiene que la expresión final del elemento de matriz $\left\langle |\mathcal{M}|^{2}\right\rangle$ es:
\begin{equation}
\left\langle |\mathcal{M}|^{2}\right\rangle=\dfrac{1}{8}\left[ f_{K}\left( \dfrac{g_w}{M_W}\right) ^{2}\right] ^{2}\left[2\left( p\cdot p_{2}\right) \left( p\cdot p_{3}\right) -p^{2}\left( p_{2}\cdot p_{3}\right)\right]\label{eq:elementomatrix}
\end{equation}

Teniendo en cuenta que $p=p_{2}+p_{3}$ junto con $p^2={m_K}^2$ y ${p_3}^2={m_{\Pl}}^2$, y sabiendo que el antineutrino carece de masa: ${p_2}^2={m_{\nu}}^2=0$, reescribimos:
\begin{equation}
p \cdot p_{2} = \left( p_{2}+ p_{3}\right) \cdot p_{2} = {p_2}^{2}+ p_{3} \cdot p_{2} \longrightarrow p\cdot p_{2}= p_{3} \cdot p_{2}
\end{equation}
\begin{equation}
p \cdot p_{3}=\left( p_{2}+ p_{3}\right) \cdot p_{3} = p_{2} \cdot p_{3}+ {p_{3}}^2 \longrightarrow p \cdot p_{3} = p_{2} \cdot p_{3}+{m_{\Pl}}^2
\end{equation}
Además,
\begin{equation}
p^{2}={p_{2}}^{2}+{p_3}^{2}+2\left( p_{2} \cdot p_{3}\right) \longrightarrow 2\left( p_{2} \cdot p_{3}\right)=\left( m_{K}^{2}-m_{\Pl}^{2}\right)
\end{equation}
Por lo tanto:
\begin{multline}
\left[ 2\left( p\cdot p_{2}\right) \left( p\cdot p_{3}\right) -p^{2}\left( p_{2}\cdot p_{3}\right) \right] = \left[ 2\left( p_{2}\cdot p_{3}\right) \left( {m_{\Pl}}^{2}+p_{2}\cdot p_{3}\right) -{m_{K}}^{2}\left( p_{2}\cdot p_{3}\right) \right] = \\ 2\left( p_{2} \cdot p_{3}\right) \left[ {m_{\Pl}}^{2}+\left( p_{2}\cdot p_{3}\right) -\dfrac{{m_K}^{2}}{2}\right] = \left({m_{K}}^{2}-{m_{\Pl}}^{2}\right) \left[{m_{\Pl}}^{2}-\dfrac{{m_K}^{2}}{2}+\dfrac{\left({m_{\PK}}^{2}-{m_{\Pl}}^{2}\right) }{2}\right] =\\ \left({m_{K}}^{2}-{m_{\Pl}}^{2}\right) \left[{m_{\Pl}}^{2}-\cancel{\dfrac{{m_{K}}^{2}}{2}}+\cancel{\dfrac{{m_{K}}^{2}}{2}}-\dfrac{{m_{\Pl}}^{2}}{2}\right] =\dfrac{{m_{\Pl}}^{2}}{2}\left( {m_{K}}^{2}-{m_{\Pl}}^{2}\right) 
\end{multline}

Y, sustituyendo en la expresión \ref{eq:elementomatrix}, se obtiene:
\begin{equation}
\left\langle |\mathcal{M}|^{2}\right\rangle=\left( \dfrac{g_w}{4M_W}\right)^{4} {f_K}^2 {m_{\Pl}}^{2}\left( {m_{K}}^{2}-{m_{\Pl}}^{2}\right)\label{eq:meanamplitude}
\end{equation}

Para obtener la probabilidad de decaimiento, aplicamos la Regla de Oro de Fermi para las desintegraciones del tipo $ 1 \rightarrow 2+3$:
\begin{equation}
d\Gamma =\left| \mathcal{M}\right| ^{2}\dfrac{S}{2m_{1}}\left[ \left( \dfrac{d^{3}\boldsymbol{p}_{2}}{\left( 2\pi \right) ^{3}2E_{2}}\right) \left( \dfrac{d^{3}\boldsymbol{p}_{3}}{\left( 2\pi \right) ^{3}2E_{3}}\right) \right] \times \left( 2\pi \right) ^{4}\delta ^{4}\left( p_{1}-p_{2}-p_{3}\right) 
\end{equation}

Donde $S$ es un producto de factores estadísticos: $\dfrac{1}{j!}$ por cada grupo de $j$ partículas idénticas en el estado final \cite{Griffiths2008}.

Integrando sobre todos los momentos, se obtiene:
\begin{equation}
\Gamma =\dfrac{S}{m_{K}}\left( \dfrac{1}{4\pi }\right) ^{2}\dfrac{1}{2}\int \dfrac{\left| \mathcal{M}\right| ^{2}}{E_{2}E_{3}}\delta ^{4}\left( p-p_{2}-p_{3}\right) d^{3}\boldsymbol{p}_{2}d^{3}\boldsymbol{p}_{3}
\end{equation}

Por simplicidad suponemos que la partícula que decae, el mesón $\PKm$ está en reposo. De este modo se tiene:
\begin{itemize}
\item $E_1=m_1=m_K$ y $\boldsymbol{p}=0$
\item El antineutrino no tiene masa $m_{\nu}=$ pero sí momento $|E_2|=\boldsymbol{p_2}$
\item El leptón sí tiene masa $m_{\Pl} \neq 0$, luego $E_3=\sqrt{m_{\Pl}+{p_{3}}^2}$
\end{itemize}

Expresando la función delta $\delta$ haciendo uso de lo anterior:
\begin{equation}
\delta ^{4}\left( p-p_{2}-p_{3}\right) =\delta \left( m_{K}-E_{2}-E_{3}\right) \delta ^{3}\left( -\boldsymbol{p}_{2}-\boldsymbol{p}_{3}\right) 
\end{equation}

Se puede reescribir la integral como:
\begin{multline}
\Gamma =\dfrac{S}{m_{K}}\dfrac{1}{2\left( 4\pi \right) ^{2}}\int \dfrac{\left|\mathcal{M}\right| ^{2}\delta ^{4}\left( p-p_{2}-p_{3}\right) }{\left| \boldsymbol{p}_{2}\right| \sqrt{{m_{\Pl}}^{2}+\boldsymbol{p}_{3} ^{2}}} d^{3}\boldsymbol{p}_{2} d^{3}\boldsymbol{p}_{3} = \\ \dfrac{S}{m_{K}}\dfrac{1}{2\left( 4\pi \right) ^{2}}\int \dfrac{\left|\mathcal{M}\right| ^{2}\delta\left(m_{K}-\boldsymbol{p}_{2}-\sqrt{{m_{\Pl}}^{2}+\boldsymbol{p}_{2} ^{2}}\right)}{ \left| \boldsymbol{p}_{2}\right| \sqrt{{m_{\Pl}}^{2}+\boldsymbol{p}_{2} ^{2}}} d^{3}\boldsymbol{p}_{2}
\end{multline}

Ahora la expresión únicamente depende de $\boldsymbol{p}_{2}$ y para resolverla introducimos las coordenadas esféricas, denotando $\left| \boldsymbol{p}_{2}\right|=\rho$. Se tiene entonces que:
\begin{align}
d^{3}\boldsymbol{p}_{2} &= \left| \boldsymbol{p}_{2}\right| ^{2}d\left|\boldsymbol{p}_{2}\right| \sin \theta d\theta d\phi & \int \sin \theta d\theta d\phi &= 4\pi
\end{align}

Por lo tanto:
\begin{multline}
\Gamma =\dfrac{S}{m_{K}}\dfrac{1}{2\left( 4\pi \right) ^{2}}\int ^{2\pi }_{0}\int ^{\pi }_{0}\int ^{\infty }_{0} \dfrac{\left|\mathcal{M}\right| ^{2}\delta\left(m_{K}-\rho-\sqrt{{m_{\Pl}}^{2}+\rho^{2}}\right)}{\cancel{\rho} \sqrt{{m_{\Pl}}^{2}+\rho^{2}}} \rho^{\cancel{2}}d{\rho}\sin \theta d\theta d\phi = \\ \dfrac{S}{8 \pi m_{K}}\int ^{\infty }_{0} \dfrac{\left|\mathcal{M}\right| ^{2}\delta\left(m_{K}-\rho-\sqrt{{m_{\Pl}}^{2}+\rho^{2}}\right)}{\sqrt{{m_{\Pl}}^{2}+\rho^{2}}} \rho d{\rho}
\end{multline}

A continuación, realizamos un cambio de variable 
\begin{equation}
E=\rho+\sqrt{{m_{\Pl}}^{2}+\rho^{2}}\label{cambioE}
\end{equation}

siendo $E$ la energía total de las partículas finales. Entonces, derivando:
\begin{equation}
\dfrac{dE}{d\rho}=\dfrac{d}{d\rho}\left( \sqrt{{m_{\Pl}}^{2}+{\rho}^{2}}+\rho\right) =\dfrac{\cancel{2}\rho}{\cancel{2}\sqrt{{m_{\Pl}}^{2}+\rho^{2}}}+1=\dfrac{\rho + \sqrt{{m_{\Pl}}^{2}+\rho^{2}}}{\sqrt{{m_{\Pl}}^{2}+\rho^{2}}}=\dfrac{E}{\sqrt{{m_{\Pl}}^{2}+\rho^{2}}}
\end{equation}

Y evaluando la delta de Dirac:
\begin{equation}
\delta \left(m_{K}-\rho-\sqrt{{m_{\Pl}}^{2}+\rho^{2}}\right)=\delta\left(m_{K}-E\right)=\delta\left(E-m_{K}\right)
\end{equation}

Finalmente, se sustituye lo anterior en la expresión de la integral, obteniéndose:
\begin{equation}
\Gamma=\dfrac{S}{8 \pi m_{K}}\int ^{\infty }_{m_K}\left| \mathcal{M}\right| ^{2}\dfrac{\rho}{E}\delta \left(E-m_{K}\right)
\end{equation}

Cuya solución es la fórmula de la probabilidad de decaimiento con $\rho=\boldsymbol{p}_2$ :
\begin{equation}
\Gamma =\dfrac{S\left| \mathcal{M}\right| ^{2}}{8\pi m_{K}^{2}}\rho\label{eq:decayratemalo}
\end{equation}

Despejando $\rho$ de la relación \ref{cambioE} y teniendo en cuenta la conservación de la energía del proceso, la energía total de los estados finales debe ser igual a la energía inicial $E=m_K$
\begin{multline}
E=\rho+\sqrt{{m_{\Pl}}^{2}+\rho^{2}} \rightarrow \left(m_{K}-\rho\right)^{2}= {m_{\Pl}}^{2}+\rho^{2} \rightarrow \\ {m_{K}}^{2}+\cancel{\rho^{2}}-2\rho m_{K}={m_{\Pl}}^{2}+\cancel{\rho^{2}} \rightarrow \rho = \dfrac{1}{2 m_K}\left({m_{K}}^{2}-{m_{\Pl}}^{2}\right)
\end{multline}

En nuestro caso, como las partículas finales son distintas, el factor $S$ es simplemente 1. Sustituyendo en \ref{eq:decayratemalo} la expresión obtenida para la amplitud de probabilidad media \ref{eq:meanamplitude}, se tiene que:
\begin{equation}
\Gamma =\dfrac{{f_{K}}^{2}}{\pi {m_{K}}^{3}}\left( \dfrac{g_w}{4M_W}\right)^{4}{m_{\Pl}}^{2}\left({m_{K}}^{2}-{m_{\Pl}}^{2}\right)^{2}\label{eq:decayratebueno}
\end{equation}

