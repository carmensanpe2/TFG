\appendix
\chapter*{Apéndice B}\label{cap:B}
\addcontentsline{toc}{chapter}{Apéndice B}
\setcounter{section}{0}
\renewcommand{\thesection}{B.\arabic{section}}
\setcounter{table}{0}
\renewcommand{\thetable}{B.\arabic{table}}
\setcounter{equation}{0}
\renewcommand{\theequation}{B.\arabic{equation}}

\section{Amplitud de transición $\mathcal{M}$ en el proceso $\PKm \rightarrow \Pl + \APnulepton$}\label{sec:detailed_calc}

El decaimiento leptónico $\PKm \rightarrow \Plm + \Pagnl$ puede observarse en la figura \ref{fig:diagrama1}. La asignación de cuadrivectores momento a cada partícula se muestra en la figura \ref{fig:diagrama2}:
\begin{itemize}
\setlength{\itemsep}{0.2\baselineskip}
\item mesón inicial $\PKm$: masa $m_K$ y cuadri-momento $p$.
\item bosón intercambiado $\PWm$: masa $M_W$ y cuadri-momento $q$.
\item antineutrino final $\Pagnl$: masa $m_{\nu}=0$ y cuadri-momento $p_2$.
\item leptón final $\Plm$: masa $m_{\Pl}$ y cuadri-momento $p_3$.
\end{itemize}

Analizamos el diagrama de la figura \ref{fig:diagrama2} y aplicamos las reglas de Feynman (presentadas al final de la sección \ref{sec:charged_kaon_decay}) para calcular la amplitud de decaimiento $\mathcal{M}$ del proceso. De las reglas 1-5 de Feynman y empleando unidades naturales, se llega a la expresión de la integral:
\begin{multline}
\int[\overline{u}\left( 3\right) \left( \dfrac{-ig_w}{2\sqrt{2}}\right) ^{2}\gamma _{\mu }\left( 1-\gamma ^{5}\right) v\left( 2\right)]\dfrac{ig{_{\mu \nu}}}{\left( M_W\right)^{2}}F^{\mu}\\ \times \left( 2\pi \right) ^{4}\delta ^{4}\left( q-p_{2}-p_{3}\right) \cancel{\left( 2\pi \right)^{4}}\delta ^{4}\left( p-q\right) \dfrac{d^{4}q}{\cancel{\left( 2\pi \right) ^{4}}} . \label{eq:intM}
\end{multline}

Con este desarrollo estamos construyendo una ecuación que nos permitirá calcular la amplitud $\mathcal{M}$. Esta integral (\ref{eq:intM}) se corresponde a la parte derecha de dicha igualdad.

Integramos como nos indica la regla 6 del cálculo de Feynman y, sabiendo que $p=q$, se obtiene lo siguiente:
\begin{equation}
[\overline{u}\left( 3\right) \left( \dfrac{-ig_w}{2\sqrt{2}}\right) ^{2}\gamma _{\mu }\left( 1-\gamma ^{5}\right) v\left( 2\right)] \dfrac{ig{_{\mu \nu}}}{\left( M_W\right)^{2}}F^{\mu}\left( 2\pi \right) ^{4}\delta ^{4}\left(p-p_{2}-p_{3}\right) .
\end{equation}

A continuación, de acuerdo con la regla 7 de Feynman, hacemos uso de la función $\delta$, $\left( 2\pi \right) ^{4}\delta ^{4}\left(p-p_{2}-p_{3}\right)$ e igualamos a $-i\mathcal{M}$ (parte izquierda) para calcular $\mathcal{M}$:

\begin{equation}
-i\mathcal{M}=[\overline{u}\left( 3\right) \left( \dfrac{-ig_w}{2\sqrt{2}}\right) ^{2}\gamma _{\mu }\left( 1-\gamma ^{5}\right) v\left( 2\right)] \dfrac{ig{_{\mu \nu}}}{\left( M_W\right)^{2}}F^{\mu} .
\end{equation}

Finalmente, se obtiene:
\begin{equation}
\mathcal{M} =\dfrac{{g_{w}}^2}{8\left( M_W\right)^{2}}\left[ \overline{u}\left(3\right) \gamma_{\mu}\left( 1-\gamma ^{5} \right) v\left( 2\right) \right] F^{\mu} .
\end{equation}
con $F^{\mu}=f_K p^{\mu}$.

Sumando sobre todos los espines posibles de las partículas finales y haciendo el promedio sobre los iniciales, se llega a esta expresión:
\begin{equation}
\left\langle |\mathcal{M}|^{2}\right\rangle=\left[ \dfrac{f_{K}}{8}\left( \dfrac{g_w}{M_W}\right)^{2}\right]^{2} p_{\mu }p_{\nu} \left[ \overline{u}\left( 3\right)\gamma^{\mu}\left( 1-\gamma^{5}\right) v\left( 2\right) \right] \left[ \overline{u}\left( 3\right) \gamma^{\nu} \left( 1-\gamma^{5}\right) v\left( 2\right) \right] ^{\ast } .
\end{equation}

Haciendo uso de los teoremas de trazas de las matrices de Dirac \ref{eq:casimir_trick}, resulta:
\begin{multline}
\left[ \overline{u}\left( 3\right)\gamma^{\mu}\left( 1-\gamma^{5}\right) v\left( 2\right) \right] \left[ \overline{u}\left( 3\right) \gamma^{\nu }\left( 1-\gamma^{5}\right) v\left( 2\right) \right] \\ \equiv \Tr \left[ \gamma^{\mu } \left(1 -\gamma ^{5}\right) \slashed{p_{2}}\gamma^{\nu }\left( 1-\gamma ^{5}\right) \left( \slashed{p_{3}}+m_{\Pl}\right) \right] .
\end{multline}

Expandiendo este producto de trazas:
\begin{multline}
\Tr \left[ \gamma^{\mu } \left(1 -\gamma ^{5}\right) \slashed{p_{2}}\gamma^{\nu }\left( 1-\gamma ^{5}\right) \left( \slashed{p_{3}}+m_{\Pl}\right) \right] = \Tr[ \gamma ^{\mu }\slashed{p_{2}}\gamma ^{\nu }\slashed{p_{3}} + \cancel{\gamma ^{\mu }\slashed{p_{2}}\gamma ^{\nu }m_{\Pl}} -\gamma ^{\mu }\slashed{p_{2}}\gamma ^{\nu }\gamma ^{5}\slashed{p_{3}} \\ - \cancel{\gamma ^{\mu }\slashed{p_{2}}\gamma ^{\nu }\gamma ^{5}m_{\Pl}} -\gamma ^{\mu }\gamma ^{5}\slashed{p_{2}}\gamma ^{\nu }\slashed{p_{3}}- \cancel{\gamma ^{\mu }\gamma ^{5}\slashed{p_{2}}\gamma ^{\nu }m_{\Pl}} +\gamma ^{\mu }\gamma ^{5}\slashed{p_{2}}\gamma ^{\nu }\gamma ^{5}\slashed{p_{3}} + \cancel{\gamma ^{\mu }\gamma ^{5}\slashed{p_{2}}\gamma ^{\nu }\gamma ^{5}m_{\Pl}} .
\end{multline}

Los términos anteriores se han tachado por la propiedad 10 de las trazas (ver Apéndice \hyperref[cap:A]{A}). Usando la propiedad 3, reescribimos como:
\begin{multline}
\Tr[ \gamma ^{\mu }\slashed{p_{2}}\gamma ^{\nu }\slashed{p_{3}} -\gamma ^{\mu }\slashed{p_{2}}\gamma ^{\nu }\gamma ^{5}\slashed{p_{3}} -\gamma ^{\mu }\gamma ^{5}\slashed{p_{2}}\gamma ^{\nu }\slashed{p_{3}} +\gamma ^{\mu }\gamma ^{5}\slashed{p_{2}}\gamma ^{\nu }\gamma ^{5}\slashed{p_{3}}] = \\ \Tr[ \gamma ^{\mu }\slashed{p_{2}}\gamma ^{\nu }\slashed{p_{3}}] - \Tr[\gamma ^{\mu }\slashed{p_{2}}\gamma ^{\nu }\gamma ^{5}\slashed{p_{3}}] - \Tr[\gamma ^{\mu }\gamma ^{5}\slashed{p_{2}}\gamma ^{\nu }\slashed{p_{3}}] + \Tr[\gamma ^{\mu }\gamma ^{5}\slashed{p_{2}}\gamma ^{\nu }\gamma ^{5}\slashed{p_{3}}] .
\end{multline}

Evaluando cada traza recordando la notación de Feynmann (eq. \ref{eq:slashnot}) y las propiedades de la sección \ref{sec:trazas}:

\begin{itemize}
\item $\begin{multlined}[t]
\Tr[ \gamma ^{\mu }\slashed{p_{2}}\gamma ^{\nu }\slashed{p_{3}}]= \left( p_{2}\right) _{\lambda }\left( p_{3}\right) _{\sigma } \Tr\left[ \gamma ^{\mu }\gamma ^{\lambda }\gamma ^{\nu }\gamma ^{\sigma }\right] = \\ \left( p_{2}\right) _{\lambda }\left( p_{3}\right) _{\sigma }4\left( g^{\mu\nu}g^{\lambda \sigma} -g^{\mu \lambda }g^{\nu \sigma }+g^{\mu \sigma }g^{\nu \lambda }\right) = 4\left[{p_{2}}^{\mu }{ p_{3}}^{\nu }-g_{\mu \nu }\left( p_{2}\cdot p_{3}\right) +{p_{3}}^{\mu } {p_{2}}^{\nu }\right]
\end{multlined}$
\item $\begin{multlined}[t]
- \Tr[\gamma ^{\mu }\slashed{p_{2}}\gamma ^{\nu }\gamma ^{5}\slashed{p_{3}}] = -\left( p_{2}\right) _{\lambda }\left( p_{3}\right) _{\sigma } \Tr\left[ \gamma ^{\mu }\gamma ^{\lambda }\gamma ^{\nu }\gamma^5 \gamma ^{\sigma }\right] = \\ -\left( p_{2}\right) _{\lambda }\left( p_{3}\right) _{\sigma } \Tr\left[\gamma^5 \gamma ^{\sigma }\gamma ^{\mu }\gamma ^{\lambda }\gamma ^{\nu }\right]=-\left( p_{2}\right) _{\lambda }\left( p_{3}\right) _{\sigma }4i\varepsilon ^{\sigma \mu \lambda \nu }=-\left( p_{2}\right) _{\lambda }\left( p_{3}\right) _{\sigma }4i\varepsilon ^{\mu \nu \lambda \sigma}
\end{multlined}$
\item $\begin{multlined}[t]
- \Tr[\gamma ^{\mu }\gamma ^{5}\slashed{p_{2}}\gamma ^{\nu }\slashed{p_{3}}]= -\left( p_{2}\right) _{\lambda }\left( p_{3}\right) _{\sigma } \Tr\left[ \gamma ^{\mu }\gamma^5 \gamma ^{\lambda }\gamma ^{\nu }\gamma ^{\sigma }\right] = \\ -\left( p_{2}\right) _{\lambda }\left( p_{3}\right) _{\sigma } \Tr\left[\gamma^5\gamma ^{\lambda }\gamma ^{\nu \gamma ^{\sigma }\gamma ^{\mu }}\right]= -\left( p_{2}\right) _{\lambda }\left( p_{3}\right) _{\sigma }4i\varepsilon ^{\lambda \nu \sigma \mu} = -\left( p_{2}\right) _{\lambda }\left( p_{3}\right) _{\sigma }4i\varepsilon ^{\mu \nu \lambda \sigma}
\end{multlined}$
\item $\begin{multlined}[t]
\Tr[ \gamma ^{\mu }\gamma ^{5}\slashed{p_{2}}\gamma ^{\nu }\gamma ^{5}\slashed{p_{3}}]= \left( p_{2}\right) _{\lambda }\left( p_{3}\right) _{\sigma } \Tr\left[ \gamma ^{\mu }\gamma ^{5}\gamma ^{\lambda }\gamma ^{\nu }\gamma ^{5}\gamma ^{\sigma }\right] = \left( p_{2}\right) _{\lambda }\left( p_{3}\right) _{\sigma } \Tr\left[\gamma ^{5}\gamma ^{5} \gamma ^{\mu }\gamma ^{\lambda }\gamma ^{\nu }\gamma ^{\sigma }\right]= \\ \left( p_{2}\right) _{\lambda }\left( p_{3}\right) _{\sigma }4\left( g^{\mu\nu}g^{\lambda \sigma} -g^{\mu \lambda }g^{\nu \sigma }+g^{\mu \sigma }g^{\nu \lambda }\right) = 4\left[{p_{2}}^{\mu }{ p_{3}}^{\nu }-g_{\mu \nu }\left( p_{2}\cdot p_{3}\right) +{p_{3}}^{\mu } {p_{2}}^{\nu }\right]
\end{multlined}$ 
\end{itemize}

Aunando los cuatro resultados anteriores de las trazas, se tiene que:
\begin{multline}
\Tr \left[ \gamma^{\mu } \left(1 -\gamma ^{5}\right) \slashed{p_{2}}\gamma^{\nu }\left( 1-\gamma ^{5}\right) \left( \slashed{p_{3}}+m_{l}\right) \right] = 4\left[{p_{2}}^{\mu }{ p_{3}}^{\nu }-g_{\mu \nu }\left( p_{2}\cdot p_{3}\right) +{p_{3}}^{\mu } {p_{2}}^{\nu }\right] \\ -4i\left( p_{2}\right) _{\lambda }\left( p_{3}\right) _{\sigma }\varepsilon ^{\mu \nu \lambda \sigma} -4i\left( p_{2}\right) _{\lambda }\left( p_{3}\right) _{\sigma }\varepsilon ^{\mu \nu \lambda \sigma} + 4\left[{p_{2}}^{\mu }{ p_{3}}^{\nu }-g_{\mu \nu }\left( p_{2}\cdot p_{3}\right) +{p_{3}}^{\mu } {p_{2}}^{\nu }\right] = \\ 8\left[{p_{2}}^{\mu }{ p_{3}}^{\nu }-g_{\mu \nu }\left( p_{2}\cdot p_{3}\right) +{p_{3}}^{\mu } {p_{2}}^{\nu } -i{p_{2}}^{\lambda} {{p_{3}}}^{\sigma} \varepsilon^{  \mu \nu \lambda \sigma} \right] .
\end{multline}

Por lo tanto, queda:
\begin{equation}
\left\langle |\mathcal{M}|^{2}\right\rangle=\left[ \dfrac{f_{K}}{\cancel{8}}\left( \dfrac{g_w}{M_W}\right)^{2}\right]^{2} p_{\mu }p_{\nu}\cancel{8}\left[{p_{2}}^{\mu }{ p_{3}}^{\nu }-g_{\mu \nu }\left( p_{2}\cdot p_{3}\right) +{p_{3}}^{\mu } {p_{2}}^{\nu } -i{p_{2}}^{\lambda} {{p_{3}}}^{\sigma} \varepsilon^{  \mu \nu \lambda \sigma} \right] .
\end{equation}

Resolviendo los productos de tensores:
\begin{multline}
p_{\mu }p_{\nu}\left[{p_{2}}^{\mu }{ p_{3}}^{\nu }-g_{\mu \nu }\left( p_{2}\cdot p_{3}\right) +{p_{3}}^{\mu } {p_{2}}^{\nu } -i{p_{2}}^{\lambda} {{p_{3}}}^{\sigma} \varepsilon^{  \mu \nu \lambda \sigma} \right] =  p_{\mu }p_{\nu}{p_{2}}^{\mu }{ p_{3}}^{\nu } + p_{\mu }p_{\nu}{p_{3}}^{\mu } {p_{2}}^{\nu } \\ - p_{\mu }p_{\nu}g_{\mu \nu }\left( p_{2}\cdot p_{3}\right)- \cancel{i p_{\mu }p_{\nu}{p_{2}}^{\lambda} {{p_{3}}}^{\sigma} \varepsilon^{  \mu \nu \lambda \sigma}} = 2\left( p\cdot p_{2}\right) \left( p\cdot p_{3}\right) -p^{2}\left( p_{2}\cdot p_{3}\right) .
\end{multline}
El término tachado es nulo por la propiedad 16.

Finalmente, se tiene que la expresión del elemento de matriz $\left\langle |\mathcal{M}|^{2}\right\rangle$ es:
\begin{equation}
\left\langle |\mathcal{M}|^{2}\right\rangle=\dfrac{1}{8}\left[ f_{K}\left( \dfrac{g_w}{M_W}\right) ^{2}\right] ^{2}\left[2\left( p\cdot p_{2}\right) \left( p\cdot p_{3}\right) -p^{2}\left( p_{2}\cdot p_{3}\right)\right] .\label{eq:elementomatrix}
\end{equation}

Teniendo en cuenta que $p=p_{2}+p_{3}$ junto con $p^2={m_K}^2$ y ${p_3}^2={m_{\Pl}}^2$, y sabiendo que el antineutrino carece de masa: ${p_2}^2={m_{\nu}}^2=0$, reescribimos:
\begin{equation}
p \cdot p_{2} = \left( p_{2}+ p_{3}\right) \cdot p_{2} = {p_2}^{2}+ p_{3} \cdot p_{2} \longrightarrow p\cdot p_{2}= p_{3} \cdot p_{2} ,
\end{equation}
\begin{equation}
p \cdot p_{3}=\left( p_{2}+ p_{3}\right) \cdot p_{3} = p_{2} \cdot p_{3}+ {p_{3}}^2 \longrightarrow p \cdot p_{3} = p_{2} \cdot p_{3}+{m_{\Pl}}^2 .
\end{equation}
Además,
\begin{equation}
p^{2}={p_{2}}^{2}+{p_3}^{2}+2\left( p_{2} \cdot p_{3}\right) \longrightarrow 2\left( p_{2} \cdot p_{3}\right)=\left( m_{K}^{2}-m_{\Pl}^{2}\right) .
\end{equation}
Por lo tanto:
\begin{multline}
\left[ 2\left( p\cdot p_{2}\right) \left( p\cdot p_{3}\right) -p^{2}\left( p_{2}\cdot p_{3}\right) \right] = \left[ 2\left( p_{2}\cdot p_{3}\right) \left( {m_{\Pl}}^{2}+p_{2}\cdot p_{3}\right) -{m_{K}}^{2}\left( p_{2}\cdot p_{3}\right) \right] = \\ 2\left( p_{2} \cdot p_{3}\right) \left[ {m_{\Pl}}^{2}+\left( p_{2}\cdot p_{3}\right) -\dfrac{{m_K}^{2}}{2}\right] = \left({m_{K}}^{2}-{m_{\Pl}}^{2}\right) \left[{m_{\Pl}}^{2}-\dfrac{{m_K}^{2}}{2}+\dfrac{\left({m_{\PK}}^{2}-{m_{\Pl}}^{2}\right) }{2}\right] =\\ \left({m_{K}}^{2}-{m_{\Pl}}^{2}\right) \left[{m_{\Pl}}^{2}-\cancel{\dfrac{{m_{K}}^{2}}{2}}+\cancel{\dfrac{{m_{K}}^{2}}{2}}-\dfrac{{m_{\Pl}}^{2}}{2}\right] =\dfrac{{m_{\Pl}}^{2}}{2}\left( {m_{K}}^{2}-{m_{\Pl}}^{2}\right) .
\end{multline}

Y, sustituyendo en la expresión \ref{eq:elementomatrix}, se obtiene:
\begin{equation}
\left\langle |\mathcal{M}|^{2}\right\rangle=\left( \dfrac{g_w}{2M_W}\right)^{4} {f_K}^2 {m_{\Pl}}^{2}\left( {m_{K}}^{2}-{m_{\Pl}}^{2}\right) .\label{eq:averageM}
\end{equation}