\appendix
\chapter*{Apéndice A}\label{appendix}
\addcontentsline{toc}{chapter}{Apéndice A}
\setcounter{section}{0}
\renewcommand{\thesection}{A.\arabic{section}}

\section{Transformaciones de Lorentz}\label{cap:Lorentz}
Las \textit{Transformaciones de Lorentz} (LT) son las expresiones matemáticas encargadas de relacionar un suceso observado desde dos sistemas de referencia distintos, en base a los postulados de la Teoría de Relatividad Especial. Son las siguientes:
\begin{align}
x' &= \dfrac{x-vt}{\sqrt{1-v^2/c^2}} & y' &=y & z' &=z & t' &=\dfrac{t-xv/c^2}{\sqrt{1-v^2/c^2}}\label{eq:TLorentz1}
\end{align}

Denotando $\beta=v/c$ y $\gamma=\sqrt{1-v^2/c^2}$, reescribimos las LT como:
\begin{align}
x' &= \gamma(x-\beta ct) & y' &=y & z' &=z & t' &=\gamma \left(t-\dfrac{\beta x}{c}\right) \label{eq:TLorentz2}
\end{align}

\section{Cuadrivectores}\label{cap:four-vectors}
La forma más sencilla de expresar las LT es utilizando los cuadrivectores: vectores de cuatro componentes que permiten apreciar facilmente qué magnitudes son invariantes frente a las LT (formulación covariante).
\begin{align}
(ct,x, y, z) &=(x^0, x^1, x^2, x^3)=X^\mu & (E,cp_x, cp_y, cp_z) &=(p^0, p^1, p^2, p^3)=P^\mu
\end{align}
Hay dos tipos de cuadrivectores:
\begin{itemize}
\item Contravariante (índice arriba): $X^\mu=(x^0, x^1, x^2, x^3)=(x^0,\textbf{x}) \rightarrow (t,\textbf{x})$
\item Covariante (índice abajo): $X_\mu=(x_0, x_1, x_2, x_3)=(x^0,-\textbf{x}) \rightarrow (t,-\textbf{x})$
\end{itemize} 
Para relacionarlos, se usa el tensor métrico $g_{\mu\nu}$ que define la métrica del espacio-tiempo de Minkowski.
\begin{align}
A_{\mu } &=\sum ^{3}_{\nu =0}g_{\mu \nu }A^{\nu} \equiv g_{\mu \nu }A^{\nu} & A^{\mu } &=\sum ^{3}_{\nu =0}g^{\mu \nu }A_{\nu} \equiv g^{\mu \nu }A_{\nu}
\end{align}
con $g_{\mu\nu} = g^{\mu\nu}$, siendo:
\begin{equation*}
g_{\mu\nu} = 
\begingroup 
\renewcommand*{\arraystretch}{0.8}
\setlength\arraycolsep{10pt}
\begin{pmatrix}
1 & & &  \\
& -1 & & \\
& & -1 & \\
& & & -1
\end{pmatrix}
\endgroup
\end{equation*}
Por lo tanto, las LT para las componentes contravariantes y covariantes, respectivamente, pueden expresarse tal que así:
\begin{align}
X'^{\mu } &=\Lambda\indices{^\mu _\nu}X^\nu & X'_{\mu } &=\Lambda\indices{_\mu ^\nu}X_\nu
\end{align}

\section{Ecuación de Dirac}\label{cap:Dirac}
Dirac quería hallar una ecuación linear en $t$ que tratase las variables espaciales y temporales en forma simétrica, que tuviera la norma conservada definida positiva (lo que implica que $\widehat{H}$ sea hermítico) y que fuera covariante.\cite{MCR} Así se llegó a esta expresión:

\begin{equation}
i\partial _{t}\psi =-i\left( \alpha _{1}\dfrac{\partial }{\partial x^{1}}+\alpha _{2}\dfrac{\partial }{\partial x^{2}} + \alpha _{3}\dfrac{\partial }{\partial x^{3}}\right) \psi +\beta M\psi \equiv\widehat{H}\psi
\end{equation}
Donde los coeficientes $\alpha_i$ y $\beta$ son matrices. En la representación más común (la de Dirac-Pauli), los coeficientes son:
\begin{align*}
\alpha _{i}=
\begingroup 
\renewcommand*{\arraystretch}{0.8}
\setlength\arraycolsep{10pt}
\begin{pmatrix} 
0 & \boldsymbol{\sigma_{i}} \\ \boldsymbol{\sigma_{i}} & 0 \end{pmatrix} \qquad
\beta =\begin{pmatrix} \mathbb{1} & 0 \\ 0 & -\mathbb{1} \end{pmatrix}
\endgroup
\end{align*}
siendo las \boldsymbol{$\sigma_{i}$} las matrices de Pauli:
\begin{align*}
\sigma _{x}=
\begingroup 
\renewcommand*{\arraystretch}{0.8}
\setlength\arraycolsep{10pt}
\begin{pmatrix} 
0 & 1 \\ 1 & 0 \end{pmatrix} \qquad \sigma_y
\begin{pmatrix} 
0 & -i \\ i & 0 \end{pmatrix} \qquad
\sigma_z =\begin{pmatrix} 1 & 0 \\ 0 & -1 \end{pmatrix}
\endgroup
\end{align*}

\subsection{Matrices y espinores de Dirac en notación covariante}\label{cap:matrix_spinors}