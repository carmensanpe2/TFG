\appendix
\chapter*{Apéndice A}\label{appendix}
\addcontentsline{toc}{chapter}{Apéndice A}
\setcounter{section}{0}
\renewcommand{\thesection}{A.\arabic{section}}

\section{Transformaciones de Lorentz}\label{cap:Lorentz}
Las \textit{Transformaciones de Lorentz} (LT) son las expresiones matemáticas encargadas de relacionar un suceso observado desde dos sistemas de referencia distintos, en base a los postulados de la Teoría de Relatividad Especial. Son las siguientes:
\begin{align}
x' &= \dfrac{x-vt}{\sqrt{1-v^2/c^2}} & y' &=y & z' &=z & t' &=\dfrac{t-xv/c^2}{\sqrt{1-v^2/c^2}}\label{eq:TLorentz1}
\end{align}

Denotando $\beta=v/c$ y $\gamma=\sqrt{1-v^2/c^2}$, reescribimos las LT como:
\begin{align}
x' &= \gamma(x-\beta ct) & y' &=y & z' &=z & t' &=\gamma \left(t-\dfrac{\beta x}{c}\right) \label{eq:TLorentz2}
\end{align}

\section{Cuadrivectores y notación covariante}\label{cap:four-vectors}
\vspace{5mm}

Los cuadrivectores o tetravectores 

\section{Ecuación de Dirac. Matrices y spinores de Dirac}\label{cap:Dirac}