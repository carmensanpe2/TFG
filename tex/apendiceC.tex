\appendix
\chapter*{Apéndice C}\label{cap:C}
\addcontentsline{toc}{chapter}{Apéndice C}
\setcounter{section}{0}
\renewcommand{\thesection}{C.\arabic{section}}
\setcounter{table}{0}
\renewcommand{\thetable}{C.\arabic{table}}
\setcounter{equation}{0}
\renewcommand{\theequation}{C.\arabic{equation}}

\section{Helicidad y Quiralidad}\label{sec:quirality}
En las expresiones anteriores de la corriente débil aparece el operador $\gamma^5$, que indica la quiralidad de las partículas. A menudo, quiralidad y helicidad son conceptos confundidos. Por este motivo, es conveniente hacer un paréntesis y explicar con detalle en qué consiste cada uno.

La helicidad se define como la proyección del espín de una partícula en la dirección de su momento $\vec{p}$. En el marco teórico de la TCC, la helicidad de la partícula se representa mediante el operador helicidad $\mathcal{H}$:
\begin{equation}
\mathcal{H}=\dfrac{1}{2} \dfrac{\vec{p} \cdot \vec{\sigma}}{p}
\end{equation} 

Para una partícula fermiónica (de espín $1/2$), los autovalores de $\mathcal{H}$ son $+1/2$ (helicidad positiva), si la dirección de su espín coincide con la dirección de su movimiento, y $-1/2$ (helicidad negativa) en el caso contrario \cite{Bettini}. Así pues, los autoestados de helicidad se corresponden a espinores de dos componentes. En general, si la partícula tiene una masa distinta de cero, la helicidad no es un invariante de Lorentz.

La quiralidad es una propiedad de los bi-espinores que se utilizan para definir partículas y se representa por los autoestados del operador $\gamma^5$, con valores propios $\pm 1$. Los estados de quiralidad son designados como positivo o derecha (R) y negativo o izquierda (L), y sus proyectores son:

\begin{align}
\psi_L &= \dfrac{1-\gamma^5}{2}\psi & \psi_R &= \dfrac{1+\gamma^5}{2}\psi
\end{align}

A diferencia de la helicidad, la quiralidad es siempre un invariante de Lorentz. Sin embargo, cuando la partícula fermiónica carece de masa, helicidad y quiralidad coinciden.
El factor $\dfrac{1-\gamma^5}{2}$ es responsable de que la interacción débil presente estructura V-A, provocando a su vez que se viole la conservación de P y C en esta interacción. Todo esto implica que sólo los leptones con quiralidad negativa (izquierda) pueden reaccionar a la fuerza débil.
