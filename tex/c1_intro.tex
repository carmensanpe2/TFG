\renewcommand{\listtablename}{Índice de tablas}
\renewcommand{\tablename}{Tabla}

\chapter{Introducción}\label{cap:intro}
El período entre 1940-1950 fue clave para el desarrollo de la Física de Partículas debido a los numerosos descubrimientos que se llevaron a cabo. Yukawa había propuesto hace unos años atrás la existencia de una partícula portadora de la Interacción Fuerte, cuya masa estuviera entre la del protón y la del electrón y denominada por este motivo mesón (middle weight).
 
En los años posteriores, los científicos no cesaron de realizar experimentos en las cámaras de niebla tratando de identificar la partícula de Yukawa. Primero se descubrió en 1936 el muón $\mu$, una partícula cuya masa coincidía con la descrita por Yukawa pero que fue descartada al comprobar que su sección eficaz no era la propia de la Interacción Fuerte. En 1947, el grupo de investigación de Powell descubrió el mesón $\pi$ o pión, y esta vez la partícula sí coincidía con las predicciones de Yukawa. Sin embargo, este no fue el único descubrimiento realizado en 1947.

Los físicos británicos Rochester y Butler, se hallaban también ese año realizando experimentos en una cámara de niebla cuando observaron unos rastros inusuales en ella que tenían forma de V invertida. Este hecho supuso la primera observación de los mesones K. Decidieron repetir el experimento en los pirineos franceses, detectando decenas de estas nuevas partículas. En 1949, el grupo de Powell logró también observar un rastro parecido que indicaba la presencia de esta nueva partícula V, que luego decaía en tres piones. 

Con el paso del tiempo, las técnicas de detección de partículas mejoraron enormemente y durante la copiosa producción de estas partículas V en los experimentos, se observó, entre otros curiosos fenómenos, que a pesar de que la Interacción Fuerte era la responsable de su formación, su larga vida indicaba que su desintegración se producía mediante Interacción Débil. Esta junto a otras características inusuales, hizo que estas partículas se ganaran el sobrenombre de extrañas. Desde entonces, se han detectado otras partículas extrañas tales como los bariones $\Sigma$ y $\Lambda^0$.

Los científicos de la época propusieron varias teorías, pero finalmente se concluyó que era necesario introducir un nuevo número cuántico para darle explicación a este suceso: la extrañeza o strangeness $S$. \cite{Bardeen2012}\cite{Griffiths2008} \\

Así pues, podemos concluir que los mesones K son hadrones de tipo mesón; son partículas bosónicas que sienten la interacción fuerte y se caracterizan por tener espín entero (nulo en este caso) y número bariónico nulo (por ser mesones). Se consideran partículas ``estables'' porque, generalmente, decaen en hadrones más ligeros mediante interacción débil en lugar de por interacción fuerte o electromagnética y tienen vidas medias relativamente largas.

Actualmente se conocen 4 tipos de mesones K distintos: las partículas $K^+$ y $K^0$ con sus respectivas antipartículas $K^-$ y $\overline{\rm K^0}$. La siguiente tabla muestra un resumen de sus propiedades:\\

\begin{table}[h]
	\centering
	\begin{tabular}{l*{7}{c}r}
\hline
Partícula & $K^+$ & $K^-$ & $K^0$ & $\overline{\rm K^0}$ \\ 
\hline
Carga $Q$ & $1$ & $-1$ & $0$ & $0$\\
Masa $(MeV/c^2)$ & $493,677$ & $493,677$ & $497,611$ & $497,611$\\
Isospín $I$ & $1/2$ & $1/2$ & $1/2$ & $1/2$ \\
3ª componente del Isoespín $I_3$ & $1/2$ & $-1/2$ & $-1/2$ & $1/2$ \\
Momento angular y paridad $J^\pi$ & $0^-$ & $0^-$ & $0^-$ & $0^-$ \\
Nº Bariónico $B$ & $0$ & $0$ & $0$ & $0$\\
Extrañeza $S$ & $1$ & $-1$ & $1$ & $-1$\\
spin $s$ & $0$ & $0$ & $0$ & $0$\\ 
\hline
	\end{tabular}
\caption{Propiedades y números cuánticos relevantes de los Mesones K.\protect\footnotemark}
\label{tab:propiedades}
\end{table}

\footnotetext{Propiedades extraídas de \cite{tanabashi}, \cite{notas2020}.}

En los siguientes capítulos se detallan, en mayor profundidad, la propiedad de la extrañeza así como los decaimientos de los distintos mesones K por Interacción Débil y sus repercusiones en la violación de simetría CP.




